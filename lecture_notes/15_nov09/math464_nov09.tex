% --------------------------------------------------------------
% This is all preamble stuff that you don't have to worry about.
% Head down to where it says "Start here"
% --------------------------------------------------------------
 
\documentclass[12pt]{article}
 
\usepackage[margin=1in]{geometry} 
\usepackage{bm} % bold in mathmode \bm
\usepackage{amsmath,amsthm,amssymb,mathtools}
\usepackage{dsfont} % for indicator function \mathds 1
\usepackage{tikz,pgf,pgfplots}
\usepackage{enumerate} 
\usepackage[multiple]{footmisc} % for an adjascent footnote
\usepackage{graphicx,float} % figures
\usepackage{centernot} % for \centernot\implies (wrapped in \nimplies)

%% set noindent by default and define indent to be the standard indent length
\newlength\tindent
\setlength{\tindent}{\parindent}
\setlength{\parindent}{0pt}
\renewcommand{\indent}{\hspace*{\tindent}}

%% some math macros
\newcommand{\norm}[1]{\left\lVert#1\right\rVert} % vector norm
\newcommand*{\vv}[1]{\vec{\mkern0mu#1}} % \vec with arrow on top
\renewcommand{\Re}{\mathfrak {Re}}
\renewcommand{\Im}{\mathfrak {Im}}
\newcommand{\R}{\mathbb R}
\newcommand{\N}{\mathbb N}
\newcommand{\Z}{\mathbb Z}
\renewcommand{\P}{\mathbb P}
\newcommand{\Q}{\mathbb Q}
\newcommand{\E}{\mathbb E}
\newcommand{\F}{\mathbb F}
\newcommand{\C}{\mathbb C}
\newcommand{\X}{\mathbb X}
\newcommand{\powerset}{\mathcal P}
\renewcommand{\L}{\mathcal L}
\newcommand{\var}{\mathrm{Var}}
\newcommand{\Var}{\mathrm{Var}}
\newcommand{\cov}{\mathrm{Cov}}
\newcommand{\Cov}{\mathrm{Cov}}
\newcommand{\gm}{\mathrm{gm~}}
\newcommand{\am}{\mathrm{am~}}
\newcommand{\trace}{\mathrm{trace~}}
\newcommand{\Trace}{\mathrm{Trace~}}
\newcommand{\rank}{\mathrm{rank~}}
\newcommand{\Rank}{\mathrm{Rank~}}
\newcommand{\Span}{\mathrm{Span~}}
\newcommand{\card}{\mathrm{card}}
\newcommand{\Card}{\mathrm{Card}}
\newcommand{\limplies}{~\Longleftarrow ~} % leftwards implies, for some reason requires spacing to mirror the formatting of \implies (and therefore \rimplies below)
\newcommand{\rimplies}{\implies} % rightwards implies (for consistency)
\newcommand{\nimplies}{\centernot\implies} % rightwards implies with line struck through
\newcommand{\indist}{\,{\buildrel \mathcal D \over \sim}\,}
\newcommand\defeq{\mathrel{\stackrel{\makebox[0pt]{\mbox{\normalfont\tiny 
	def}}}{=}}} % equal sign with def above
\newcommand{\cl}{\mathrm{cl}} % closure of a set cl(X)
\newcommand{\Cl}{\mathrm{Cl}} % closure of a set Cl(X)
\newcommand{\closure}{\mathrm{closure}} % closure of a set closure(X)
\newcommand{\Closure}{\mathrm{Closure}} % closure of a set Closure(X)

\begin{document}
 
% --------------------------------------------------------------
%                         Start here
% --------------------------------------------------------------
 
\title{Real Analysis\\Lecture Notes}
\author{Metric Spaces}
\date{November 9 2016 \\ Last update: \today{}}
\maketitle

\section{Compactness}

%
% Goal
%
{\bf Goal}: Last time we introduced our goal to {\em characterize compact metric spaces} in greater detail. We will soon make use of a fairly involved proof by contrapositive. Recall that this method of proof uses the logical identity
\begin{equation*}
	\left( p \implies q \right) \iff \left( \neg q \implies \neg p \right)
\end{equation*}

In particular, let
\begin{align*}
	p ~&:~ \{O_i\} \text{ cover $X$} \\
	q ~&:~ \text{ finitely many $\{O_i\}$ cover $X$}
\end{align*}

\indent We said that a set is compact in some space of all its covers have a finite subcover. In this case we may express this definition as
\begin{equation*}
	\text{Compactness} \iff \left( p \implies q \right)
\end{equation*}

Negating these statements yield 
\begin{align*}
	\neg q ~&:~ \text{ finitely many $\{O_i\}$ {\em do not} cover $X$ } \\
	\neg p ~&:~ \text{ $\{O_i\}$ {\em do not} cover $X$ }
\end{align*}

Note that if $\{O_i\}$ {\em do not cover} $X$ then $X \nsubseteq \{O_i\}$ which is equivalent to
\begin{align*}
	\{O_i\} &\text{ do not cover $X$} \\
	\iff X &\nsubseteq \bigcup_i O_i \\
	\implies X^c &\nsupseteq \left( \bigcup_i O_i \right)^c \\
	\implies \left(X \cap X^c\right) &\nsupseteq \left( X \cap \left( \bigcup_i O_i \right)^c \right) \\
	\implies \emptyset &\nsupseteq X \setminus \left( \bigcup_i O_i \right) \\
	\implies X \setminus \left( \bigcup_i O_i \right) &\neq \emptyset \\
	\implies X \cap \left( \bigcup_i O_i \right)^c &\neq \emptyset \\
	\implies X \cap \left( \bigcap_i O_i^c \right) &\neq \emptyset \\
	\implies \bigcap_i \left( X \cap O_i^c \right) &\neq \emptyset \\
	\implies \bigcap_i \left( X \setminus O_i \right) &\neq \emptyset 
\end{align*}

\indent Note that the set defined by $\left( X \setminus O_i \right)$ is a closed set since $O_i$ is open. Therefore, we may express our definition of compactness
\begin{align*}
	p ~&:~ \{O_i\} \text{ cover $X$} \iff X \subset \bigcup_{i\in I} O_i \\
	q ~&:~ \text{ finitely many $\{O_i\}$ cover $X$} \iff X \subset \bigcup^n_{i = 1} O_i
\end{align*}

to the contrapositive, $\neg q \implies \neg p$, by
\begin{align*}
	\neg q ~&:~ \bigcap^n_{i = 1} \left( X \setminus O_i \right) \neq \emptyset \\
	\neg p ~&:~ \bigcap_{i \in I} \left( X \setminus O_i \right) \neq \emptyset 
\end{align*}

for $O_i$ open sets and $\left( X \setminus O_i \right)$ closed sets. \\

\indent Therefore, we may reformulate our definition of compactness as follows: A set $X$ is said to be \underline{compact} if, for any family of {\em closed sets} $\left( X \setminus O_i \right)$, 
\begin{equation*}
	\bigcup^n_{i = 1} \left( X \setminus O_i \right) = \emptyset \implies \bigcup_{i \in I} \left( X \setminus O_i \right) = \emptyset
\end{equation*} 

\indent That is, in some compact set $X$, the intersection of any finite number of closed sets is empty implies that the intersection of {\em all closed sets} $\left( X \setminus O_i \right)$ in the family is empty. \\

\indent Our original definition of compactness made use of open sets. Now, our current translation of compactness permits us to work with closed sets. We summarize this more succinctly by the {\em finite intersection property}: \\

%
% Definition
%
{\bf Definition: {\em (Finite intersection property)}.} A family of sets $\{M_i\}_{i\in I}$ is said to have the \underline{finite intersection property} if its intersection over any finite subcollection is nonempty:
\begin{equation*}
	\bigcap_{i \in J} M_i \quad \text{$J$ a finite subset of $I$, $J \subset I$}
\end{equation*} 

Thus, \\

%
% Definition
%
{\bf Definition: {\em (Compactness)}.} A set $X$ is said to be \underline{compact} if {\em any} family of closed subsets $\{F_i\}_{i \in I}$ with the finite intersection property itself has a nonempty intersection. \\

%
% Definition
%
{\bf Definition: {\em (Limit point)}.} A point $x$ is a \underline{limit point} of sequence the $(x_n)$ if the sequence $(x_n)$ is a Cauchy sequence towards $x$. That is,
\begin{equation*}
	\forall\,\epsilon > 0,~\exists\,N\in\N,~\forall\,n\geq N,~ |x - x_n| < \epsilon
\end{equation*}

\indent Note that this definition of a limit point {\em does not} include points in alternating sequences with finite $\limsup$ and $\liminf$. That is, consider the sequence $(x_n)$ given by
\begin{equation*}
	(x_n) = \left( \frac{1}{2}, \frac{1}{2}, \frac{2}{3}, \frac{1}{3}, \frac{3}{4}, \frac{1}{4}, ... \right)
\end{equation*}

In particular, each element in $x_n$ is defined by
\begin{equation*}
	x_n = 
	\begin{cases}
		\frac{1}{n/2 + 1} & \text{if $n$ even} \\
		\frac{(n - 1)/2 + 1}{ (n - 1)/2 + 2 } & \text{if $n$ odd}
	\end{cases} \quad n = 1, 2, ...
\end{equation*}

so that
\begin{align*}
	\limsup (x_n) &= 1 \\
	\liminf (x_n) &= 0
\end{align*}

\indent For this definition of $(x_n)$ we see that both $x = 1$ and $x = 0$ are {\em similar} to limit points, but fail to satisfy the definition because it is not the case that
\begin{equation*}
	\forall\,n\geq N,~|x - x_n| < \epsilon
\end{equation*}

since the distance $|x - x_n|$ will alternate between $<\epsilon$ and $<1 - \epsilon$. For this reason we introduce a weaker notion than a limit point to describe such cases: \\

%
% Definition
% 
{\bf Definition: {\em (Cluster point)}.} The definition of a cluster point contrasts with that of a limit point in that its definition relies on the use of $\limsup$ and $\liminf$. We say that $x$ is a \underline{cluster point} of the sequence $(x_n)$ if
\begin{equation*}
	\forall\,\epsilon > 0,~\exists\,N\in\N,~\exists\,n\geq N,~ |x - x_n| < \epsilon
\end{equation*}

Note the change of the final quantifier
\begin{equation*}
	\exists\,n\geq N,~ |x - x_n| < \epsilon
\end{equation*}

in which we may ignore any future points beyond $n$ that do not satisfy the inequality. \\

Using our above example $(x_n) = \left( \frac{1}{2}, \frac{1}{2}, \frac{2}{3}, \frac{1}{3}, \frac{3}{4}, \frac{1}{4}, ... \right)$ we found that
\begin{align*}
	\limsup (x_n) &= 1 \\
	\liminf (x_n) &= 0
\end{align*}

so that both $1$ and $0$ are cluster points of $(x_n)$. If we split $(x_n)$ into the two subsequences
\begin{align*}
	\left( x_{n_{2k - 1}} \right) &= \left( \frac{1}{2}, \frac{2}{3}, \frac{3}{4}, ... \right) \to 1 \\
	\left( x_{n_{2k}} \right) &= \left( \frac{1}{2}, \frac{1}{3}, \frac{1}{4}, ... \right) \to 0
\end{align*}

we may be able to see the result that
\begin{equation*}
	\text{cluster point $\iff$ limit point of some subsequence}
\end{equation*}

Recall the Bolzano-Weierstrass property: \\

%
% Definition
%
{\bf Definition: {\em (Bolzano-Weierstrass Property)}.} We say that some set $X$ is said to have the \underline{Bolzano-Weierstrass Property} if every infinite sequence $(x_n) = x_1, x_2, x_3, ...$ taken from $X$ has a cluster point (finite $\limsup$ or $\liminf$?). \\

%
% Example
% 
{\bf Example:} $X = \R$ {\em does not} have the Bolzano-Weierstrass property since the sequence $x_n = 1, 2, 3, ...$ from $\R$ has no cluster point. \\

%
% Theorem
% 
{\bf Theorem: {\em If $X$ is compact then $X$ has the Bolzano-Weierstrass Property}.}

\begin{proof} Let $X$ be some compact set and take an infinite sequence $(x_n)$ from $X$. We want to find a cluster point of $(x_n)$. \\

Define some subsets of $X$ by
\begin{align*}
	B_1 &= \{x_1, x_2, x_3, ...\} \\
	B_2 &= \{x_2, x_3, x_4, ...\} \\
	B_3 &= \{x_3, x_4, x_5, ...\} \\
	&\vdots \\
	B_n &= \{x_n, x_{n + 1}, x_{n + 2}, ...\} 
\end{align*}

so that 
\begin{equation*}
	B_1 \supset B_2 \supset B_3 \supset \cdots \supset B_n \supset \cdots
\end{equation*}

Then their closures $\overline{B}_1, \overline{B}_2, \overline{B}_3, ...$ also satisfy
\begin{equation*}
	\overline{B}_1 \supset \overline{B}_2 \supset \overline{B}_3 \supset \cdots \supset \overline{B}_n \supset \cdots
\end{equation*}

Consider the family of these closures 
\begin{equation*}
	\left\{ \overline{B}_n ~:~ n = 1, 2, ... \right\}
\end{equation*}

\indent Clearly our family $\left\{ \overline{B}_n \right\}$ is a closed subset of $X$ by construction, so does this family $\left\{ \overline{B}_n \right\}$ have the finite intersection property? Well, for any finite intersection
\begin{equation*}
	\bigcap_{i} \overline{B}_{n_i}
\end{equation*}

there must be some greatest $m = \max \left( n_1, n_2, ..., n_k \right)$. But recall that
\begin{equation*}
	\overline{B}_1 \supset \overline{B}_2 \supset \overline{B}_3 \supset \cdots \supset \overline{B}_n \supset \cdots
\end{equation*}

Then for this maximal $m$ we have
\begin{equation*}
	\overline{B}_m = \bigcap_{i} \overline{B}_{n_i}
\end{equation*}

\indent Clearly $\overline{B}_m \neq \emptyset$ since $x_m \in B_m \in \overline{B}_m$. However, $X$ is a compact set by assumption. Therefore, for any family of sets from $X$ with the finite intersection property must itself has a nonempty intersection. Therefore, since $\overline{B}_m = \bigcap_{i} \overline{B}_{n_i}$ we have
\begin{equation*}
	\bigcap^\infty_{i = 1} \overline{B}_i \neq \emptyset
\end{equation*}

\indent Take some $x \in \bigcap^\infty_{i = 1} \overline{B}_i$. We claim that $x$ is a cluster point of our sequence $(x_n)$. In particular, take any open set $O$ of $X$ containing $x$, and take any $N$ so that $x \in \overline{B}_N$, where
\begin{equation*}
	B_N = \left\{ x_N, x_{N + 1}, x_{N + 2}, ... \right\}
\end{equation*}

Since
\begin{equation*}
	\left\{x_1, x_2, x_3, ..., x_N, x_{N + 1}, x_{N + 2}, ... \right\}
\end{equation*}

has $x$ as its cluster point, then the intersection
\begin{equation*}
	O \cap \left\{ x_N, x_{N + 1}, x_{N + 2}, ... \right\} \neq \emptyset
\end{equation*}

\indent Therefore, there exists some $n \geq N$ so that $x_n \in O$, and so $x$ is indeed a cluster point of $(x_n)$. That is,
\begin{equation*}
	\forall\,\epsilon > 0,~\exists\,N\in\N,~\exists\,n\geq N,~ |x - x_n| < \epsilon
\end{equation*}

and so since $(x_n)$ was an arbitrary infinite sequence from $X$ we may conclude that $X$ does indeed have the Bolzano-Weierstrass property, as desired.
\end{proof} \hfill

Up until now we have proven the following results
\begin{align*}
	\text{$X$ is closed-bounded} &\implies \text{$X$ is compact set} \\
	&\implies \text{Infinite sequences from $X$ has a cluster point} \\
	&\iff \text{$X$ has the Bolzano-Weierstrass property}
\end{align*}

\end{document}