% --------------------------------------------------------------
% This is all preamble stuff that you don't have to worry about.
% Head down to where it says "Start here"
% --------------------------------------------------------------
 
\documentclass[12pt]{article}
 
\usepackage[margin=1in]{geometry} 
\usepackage{bm} % bold in mathmode \bm
\usepackage{amsmath,amsthm,amssymb,mathtools}
\usepackage{dsfont} % for indicator function \mathds 1
\usepackage{tikz,pgf,pgfplots}
\usepackage{enumerate} 
\usepackage[multiple]{footmisc} % for an adjascent footnote
\usepackage{graphicx,float} % figures
\usepackage{centernot} % for \centernot\implies (wrapped in \nimplies)

%% set noindent by default and define indent to be the standard indent length
\newlength\tindent
\setlength{\tindent}{\parindent}
\setlength{\parindent}{0pt}
\renewcommand{\indent}{\hspace*{\tindent}}

%% some math macros
\newcommand{\norm}[1]{\left\lVert#1\right\rVert} % vector norm
\newcommand*{\vv}[1]{\vec{\mkern0mu#1}} % \vec with arrow on top
\renewcommand{\Re}{\mathfrak {Re}}
\renewcommand{\Im}{\mathfrak {Im}}
\newcommand{\R}{\mathbb R}
\newcommand{\N}{\mathbb N}
\newcommand{\Z}{\mathbb Z}
\renewcommand{\P}{\mathbb P}
\newcommand{\Q}{\mathbb Q}
\newcommand{\E}{\mathbb E}
\newcommand{\F}{\mathbb F}
\newcommand{\C}{\mathbb C}
\newcommand{\X}{\mathbb X}
\newcommand{\powerset}{\mathcal P}
\renewcommand{\L}{\mathcal L}
\newcommand{\var}{\mathrm{Var}}
\newcommand{\Var}{\mathrm{Var}}
\newcommand{\cov}{\mathrm{Cov}}
\newcommand{\Cov}{\mathrm{Cov}}
\newcommand{\gm}{\mathrm{gm~}}
\newcommand{\am}{\mathrm{am~}}
\newcommand{\trace}{\mathrm{trace~}}
\newcommand{\Trace}{\mathrm{Trace~}}
\newcommand{\rank}{\mathrm{rank~}}
\newcommand{\Rank}{\mathrm{Rank~}}
\newcommand{\Span}{\mathrm{Span~}}
\newcommand{\card}{\mathrm{card}}
\newcommand{\Card}{\mathrm{Card}}
\newcommand{\limplies}{~\Longleftarrow ~} % leftwards implies, for some reason requires spacing to mirror the formatting of \implies (and therefore \rimplies below)
\newcommand{\rimplies}{\implies} % rightwards implies (for consistency)
\newcommand{\nimplies}{\centernot\implies} % rightwards implies with line struck through
\newcommand{\indist}{\,{\buildrel \mathcal D \over \sim}\,}
\newcommand\defeq{\mathrel{\stackrel{\makebox[0pt]{\mbox{\normalfont\tiny 
	def}}}{=}}} % equal sign with def above
\newcommand{\cl}{\mathrm{cl}} % closure of a set cl(X)
\newcommand{\Cl}{\mathrm{Cl}} % closure of a set Cl(X)
\newcommand{\closure}{\mathrm{closure}} % closure of a set closure(X)
\newcommand{\Closure}{\mathrm{Closure}} % closure of a set Closure(X)

\begin{document}
 
% --------------------------------------------------------------
%                         Start here
% --------------------------------------------------------------
 
\title{Real Analysis\\Lecture Notes}
\author{Metric Spaces}
\date{October 31 2016 \\ Last update: \today{}}
\maketitle

\section{Distance in Metric Spaces}

\indent What is a reasonable way of defining distance in some space? Suppose we are given some set $A$ and points $x \notin A$ and $a \in A$. Clearly, under the definition of a metric we have
\begin{equation*}
	\rho(x, a) \geq 0
\end{equation*}

\indent However, what is a good way of defining the distance from $x$ to the entire set $A$? We typically evaluate this as
\begin{equation*}
	\rho(x, A) = \inf_{a\in A} \rho(x,a)
\end{equation*}

Suppose now that $A$ is a circle given by
\begin{equation*}
	A = \{(x,y) ~:~ x^2 + y^2 \leq 4\}
\end{equation*}

Then, under our definition for $\rho(x,A)$ we have
\begin{equation*}
	\rho(\text{(origin)}, A) = \rho((0,0), A) = 0
\end{equation*}

Suppose $A = [1,2) \subset \R$. We find that
\begin{align*}
	\rho(3, A) &= 1 \\
	\rho(2, A) &= 0
\end{align*}

\indent Note that $\rho(2, A) = 0$ despite the fact that $\rho(2, a \in A) > 0$ since our distance $\rho(2, A)$ is given by the infimum $\inf_{a \in A} \rho(2, a)$. \\

Recall the $\sup$ metric between functions given by
\begin{equation*}
	\rho(f,g) = \sup_{x\in D} |f(x) - g(x)|	
\end{equation*}

and consider the open ``{\em sphere}'' $B_{\sin x, \frac{1}{2}, \sup \text{metric} }$. This sphere defines a cloud around the function $\sin x$ with open boundaries given by $\sin x \pm \frac{1}{2}$. In particular, any function within these bounds will lie within the open sphere $B_{\sin x, \frac{1}{2}, \sup \text{metric} }$. \\

%
% Example
%
{\bf Example:} Let $X$ be a metric space and let $A \subset X$, $A \neq \emptyset$. Let $d$ be a metric on $X$ and define the distance from set $A$ by
\begin{equation*}
	d(x, A) = \inf_{a\in A} d(x, a) = f(x)
\end{equation*}

%
% Claim
% 
{\bf Claim: {\em $\bm f$ is continuous}.} That is, $f:A\to\R^+$ given by $f(x) = d(x,A) = \inf_{a\in A} d(x,a)$ is continuous.

\begin{proof} For all $a \in A$ and $\forall\,x,y \in X$ the distance $f(y)$ is bound above by
\begin{equation*}
	f(y) = d(x,A) = \in_{a\in A} d(x,a) \leq d(y, a)
\end{equation*}

by the definition of the infimum. Using the triangle inequality gives us
\begin{align*}
	f(y) &\leq d(y, a) \\
	&\leq d(y, x) + d(x, a) 
\end{align*}

Taking the infimum over $A$ of both sides yields
\begin{align*}
	\inf_{a \in A} f(y) &= \inf_{a\in A} d(a, A) \\
	&= \inf_{a\in A} \left[ \inf_{a \in A} d(x, a) \right] \\
	&= \inf_{a \in A} d(x, a) \\
	&= f(y) \\
	\inf_{a \in A} d(y, x) &= d(y, x) \\
	\inf_{a \in A} d(x, a) &= d(x, A) \\
	&= f(x) 
\end{align*}

Therefore, we may express our 
\begin{equation*}
	f(y) \leq d(y, x) + d(x, a)
\end{equation*}

as
\begin{align*}
	f(y) &\leq d(y, x) + f(x) \\
	\implies f(y) - f(x) &\leq d(y, x) \\
	\iff f(x) - f(y) &\leq d(x, y) \\
	\implies |f(x) - f(y)| &\leq d(x, y)
\end{align*}

Hence, $\forall\,\epsilon > 0$, take the distance between points $x$ and $y$ to be bound above by $\delta = \epsilon$ so that $d(x, y) < \delta = \epsilon$. That is,
\begin{equation*}
	\forall\,\epsilon > 0,~\exists\,\delta > 0, |x - y| < \delta \implies |f(x) - f(y)| \leq d(x, y) < \delta = \epsilon
\end{equation*}

which is precisely the definition of continuity, as desired.
\end{proof}

%
% Proposition
%
{\bf Proposition: {\em A subspace of a metric space is a metric space.}} Consider some metric space $X$ with metric $d$ and some subspace $S\subset X$. Is the topology on $S$ given by $(S, d)$ also a metric space? Yes!

\begin{proof} Recall the basic properties of a metric $d$, for all $x,y,z \in X$:
\begin{enumerate}
	\item $d(x, y) \geq 0$ 
	\item $d(x, y) = d(y, x)$ 
	\item $d(x, y) = 0 \iff x = y$
	\item $d(x, z) \leq d(x, y) + d(y, z)$
\end{enumerate}

\indent Therefore, if we take points $x, y, z \in S \subset X$ we see that all four properties hold under the metric space $(X,d)$ by assumption. So, since $x, y, z$ were arbitrary points from $S$ we have that all four properties of a metric space $(S, d)$ must be satisfied by inheritance from $X$.
\end{proof}

\indent Although we have just shown that $S \subset X$ inherits its metric from its superspace space $X$, potential ambiguities arise if we consider subtopologies $(E, d)$ and $(S, d)$ such that $E \subset S \subset X$. That is, when considering subspaces $E \subset S \subset X$ a natural question to ask is: How can we relate being {\em close in $X$} with being {\em closed in S}? \\

%
% Example
%
{\bf Example:} Let $X = \R$ and let subspaces $S$ and $E$ be given by
\begin{align*}
	S = (0, 1) \\
	E = \left( 0, \frac{1}{2} \right)
\end{align*}

so that $E \subset S \subset X$. Clearly, the closure of $E$ in $\mathbb R$ is
\begin{equation*}
	\cl(E) = \overline{E} = \left[ 0, \frac{1}{2} \right] 
\end{equation*}

However, the closure of $E$ in $S$ must be
\begin{equation*}
	\cl_S(E) = \overline{E}_S = \left( 0, \frac{1}{2} \right]
\end{equation*}

%
% Proposition
% 
{\bf Proposition:} Let $X$ be a metric space and $E$, $S$ be subspaces $E \subset S \subset X$. The {\em closure of $E$ relative to $S$} is
\begin{equation*}
	\cl_S(E) = \overline{E}_S = \overline{E}\cap S
\end{equation*}

where $\overline{E}$ is the {\em closure of $E$ relative to the common parent space $X$}. We say that subspace $A \subset S$ is {\em closed} in $S$ if
\begin{equation*}
	A = S \cap F
\end{equation*}

for some closed set $F$ which is closed in $X$. Analogously, subspace $A \subset S$ is {\em open} in $S$ if
\begin{equation*}
	A = S \cap O
\end{equation*}

for some open set $O$ which is open in $X$.

Let $A$ be any subset of $X$, $A \subset X$ and $S$ a subspace of $X$, $S \subset X$. To generate a subset of $S$ we may perform the intersection $A \cap S$. That is, if $A \subset X$ then
\begin{equation*}
	A \cap S \subset S
\end{equation*}

If we want to produce open sets in $S$ then suppose sets $O_i \subset X$ are open in $X$ so that
\begin{equation*}
	O_i \cap S \subset S
\end{equation*}

\indent In the next set of notes we will prove that by considering the intersection $O_i \cap S$ we will see that $S$ inherits the relative topology from that defined in $X$. \\

%
% Example
%
{\bf Example:} Let $X = \R$ and $S = \Q$, so that we are considering the subspace $\Q \subset \R$. What happens if we consider the intersection with the intervals $[e, \pi]$ and $(e, \pi)$? Using our previous proposition we must conclude that these intersections 
\begin{align*}
	[e,\pi] &\cap \Q  \quad \text{is {\em closed} in $\Q$, but} \\
	(e, \pi) &\cap \Q \quad \text{is {\em open} in $\Q$!}
\end{align*}

since the intersection with an open set is open and the intersection with a closed set is closed. However, since $e \notin \Q$ and $\pi notin \Q$ we find
\begin{equation*}
	[e,\pi] \cap \Q = (e, \pi)
\end{equation*}

\indent Therefore, we must conclude that , unlike the set of reals $\R$, the set of rationals $\Q$ contains sets that are {\em both closed and open}!











































\end{document}