% --------------------------------------------------------------
% This is all preamble stuff that you don't have to worry about.
% Head down to where it says "Start here"
% --------------------------------------------------------------
 
\documentclass[12pt]{article}
 
\usepackage[margin=1in]{geometry} 
\usepackage{bm} % bold in mathmode \bm
\usepackage{amsmath,amsthm,amssymb,mathtools}
\usepackage{dsfont} % for indicator function \mathds 1
\usepackage{tikz,pgf,pgfplots}
\usepackage{enumerate} 
\usepackage[multiple]{footmisc} % for an adjascent footnote
\usepackage{graphicx,float} % figures
\usepackage{centernot} % for \centernot\implies (wrapped in \nimplies)

%% set noindent by default and define indent to be the standard indent length
\newlength\tindent
\setlength{\tindent}{\parindent}
\setlength{\parindent}{0pt}
\renewcommand{\indent}{\hspace*{\tindent}}

%% some math macros
\newcommand{\norm}[1]{\left\lVert#1\right\rVert} % vector norm
\newcommand*{\vv}[1]{\vec{\mkern0mu#1}} % \vec with arrow on top
\renewcommand{\Re}{\mathfrak {Re}}
\renewcommand{\Im}{\mathfrak {Im}}
\newcommand{\R}{\mathbb R}
\newcommand{\N}{\mathbb N}
\newcommand{\Z}{\mathbb Z}
\renewcommand{\P}{\mathbb P}
\newcommand{\Q}{\mathbb Q}
\newcommand{\E}{\mathbb E}
\newcommand{\F}{\mathbb F}
\newcommand{\C}{\mathbb C}
\newcommand{\X}{\mathbb X}
\newcommand{\powerset}{\mathcal P}
\renewcommand{\L}{\mathcal L}
\newcommand{\var}{\mathrm{Var}}
\newcommand{\Var}{\mathrm{Var}}
\newcommand{\cov}{\mathrm{Cov}}
\newcommand{\Cov}{\mathrm{Cov}}
\newcommand{\gm}{\mathrm{gm~}}
\newcommand{\am}{\mathrm{am~}}
\newcommand{\trace}{\mathrm{trace~}}
\newcommand{\Trace}{\mathrm{Trace~}}
\newcommand{\rank}{\mathrm{rank~}}
\newcommand{\Rank}{\mathrm{Rank~}}
\newcommand{\Span}{\mathrm{Span~}}
\newcommand{\card}{\mathrm{card}}
\newcommand{\Card}{\mathrm{Card}}
\newcommand{\limplies}{~\Longleftarrow ~} % leftwards implies, for some reason requires spacing to mirror the formatting of \implies (and therefore \rimplies below)
\newcommand{\rimplies}{\implies} % rightwards implies (for consistency)
\newcommand{\nimplies}{\centernot\implies} % rightwards implies with line struck through
\newcommand{\indist}{\,{\buildrel \mathcal D \over \sim}\,}
\newcommand\defeq{\mathrel{\stackrel{\makebox[0pt]{\mbox{\normalfont\tiny 
	def}}}{=}}} % equal sign with def above
\newcommand{\cl}{\mathrm{cl}} % closure of a set cl(X)
\newcommand{\Cl}{\mathrm{Cl}} % closure of a set Cl(X)
\newcommand{\closure}{\mathrm{closure}} % closure of a set closure(X)
\newcommand{\Closure}{\mathrm{Closure}} % closure of a set Closure(X)

\begin{document}
 
% --------------------------------------------------------------
%                         Start here
% --------------------------------------------------------------
 
\title{Real Analysis\\Lecture Notes}
\author{Metric Spaces}
\date{November 16 2016 \\ Last update: \today{}}
\maketitle

\section{Compactness}

\indent We continue our work on expanding the properties of compact sets and its relationships to other definitions. In particular, we have found so far that
\begin{align*}
	\text{$X$ is closed-bounded} &\implies \text{$X$ is compact set} \stackrel{\text{def}}{\iff} \text{All open covers of $X$ have a finite subcover}\\
	&\implies \text{Infinite sequences from $X$ has a cluster point} \\
	&\stackrel{\text{def}}{\iff} \text{$X$ has the Bolzano-Weierstrass property} \\
	&\iff \text{$X$ is sequentially compact} \\
	&\stackrel{\text{def}}{\iff} \text{Every infinite sequence has a convergent subsequence}
\end{align*}

\indent We then reintroduced some definitions and properties of continuous functions and bounded sets. We now move on to a new definition: \\

%
% Definition
%
{\bf Definition: {\em (Totally bounded space)}.} A metric space $X$ is \underline{totally bounded} if for any $\epsilon > 0$ there are only finitely many points $x_1, x_2, ..., x_n \in X$ satisfying
\begin{equation*}
	X \subset \bigcup^n_{i = 1} S_{x_i,\epsilon} 
\end{equation*}

That is, $X$ is \underline{totally bounded} if $X$ is the union of finitely many open spheres of radius $\epsilon$. \\

%
% Example
%
{\bf Example:} Consider the bounded set $[2,3]$. Note that $[2,3]$ is compact by the Heine-Borel Theorem since it is closed and bounded. Therefore, for all open covers there is a corresponding finite subcover. Thus, for the open cover $\left\{S_{x,\epsilon}\right\}_{x\in [2,3]}$ we may produce the finite subcover $\left\{S_{{x_i},\epsilon}\right\}^n_{i = 1}$ so that
\begin{equation*}
	X \subset \bigcup^n_{i = 1} S_{{x_i},\epsilon}
\end{equation*}

and so $[2,3]$ is totally compact by definition. \\

%
% Example
%
{\bf Example:} In general, for all sets $[a,b]$ we find that $[a,b]$ is compact by the Heine-Borel Theorem. For all $x \in [a,b]$ let $O_x$ be the open interval
\begin{equation*}
	O_x = (x - \epsilon, x + \epsilon)
\end{equation*}

then the family of $O_x$ given by $\{O_x ~:~ x \in [a,b]\}$ covers our interval $[a,b]$. However, $[a,b]$ is compact. Therefore, all open covers of $[a,b]$ have some finite subcover $\{O_{x_1}, O_{x,2}, ...\}$ so that
\begin{equation*}
	[a,b] \subset \bigcup^n_{i = 1} O_{x_i} = \bigcup^n_{i = 1} (x_i - \epsilon, x_i + \epsilon)
\end{equation*}

and so our bounded set $[a,b]$ is totally compact. \\

%
% Proposition
%
{\bf Proposition: {\em Compactness $\implies$ totally bounded}.}
\begin{proof} By the same argument as our example above we find that a metric space $X$ is compact $\implies X$ is totally bounded.
\end{proof} 

%
% Example
%
{\bf Example:} Let $X = \R$ under the discrete metric
\begin{equation*}
	\rho_d(x,y) =
	\begin{cases}
		1 & \text{if $x \neq y$} \\
		0 & \text{else}
	\end{cases}
\end{equation*}

\indent We have found that such a metric space is bounded since the distance between points is always less than $M = 2$. However, is this $X = \R$ totally bounded? Take $\epsilon = \frac{1}{2}$ and note that under $\rho_d$ the sphere defined by $S_{x,\frac{1}{2}}$ is
\begin{equation*}
	S_{x,\frac{1}{2}} = \{x\}
\end{equation*}

Therefore any finite union of such spheres $S_{x_i,\frac{1}{2}}$ will be
\begin{equation*}
	\bigcup^n_{i = 1} S_{x_i,\frac{1}{2}} = \{x_i\}^n_{i = 1} \neq \R
\end{equation*}

\indent That is, since $n$ was arbitrary, we have shown that no finite set of open spheres on $\R$ in the discrete metric will be able to cover $\R$, and so $\R$ under $\rho_d$ is {\em not} totally bounded. \\

\indent This example gives us a sense of the distinction between {\em boundedness}, which $\R$ under $\rho_d$ clearly is, and {\em total boundedness}, which $\R$ under $\rho_d$ has shown to not be. Under the standard Euclidean metric $\rho(x,y) = |x - y|$ there is no such distinction in $\R$, but we have just seen at least one metric in which such a difference emerges. \\

%
% Proposition
%
{\bf Proposition: {\em Total boundedness $\implies$ boundedness}.} 

\begin{proof} For boundedness we require $\sup \rho(x,y) \leq M$ for all $x,y \in X$. Let $\epsilon = 1$. Then, assuming $X$ is totally bounded, there exists finitely many points $x_1, x_2, ..., x_n$ such that
\begin{equation*}
	X \subset \bigcup^n_{i = 1} S_{x_i, 1}
\end{equation*}

Now, take any two points $y_1, y_2 \in X$ such that
\begin{align*}
	y_1 &\in S_{x_1, 1} \\
	y_2 &\in S_{x_2, 1}
\end{align*}

By the definition of a metric we have the inequality
\begin{align*}
	\rho(y_1, y_2) &\leq \rho(y_1, x_{n_1}) + \rho(x_{n_1}, x_{n_2}) + \rho(x_{n_2}, y_2) \\
	&= 1 + \rho(x_{n_1}, x_{n_2}) + 1 \\
	&= \rho(x_{n_1}, x_{n_2}) + 2
\end{align*}

\indent Although we don't know the distance $\rho(x_{n_1}, x_{n_2})$ we do know that there are only finitely many such points. Thus, take the largest such distance
\begin{equation*}
	\sup \left\{ \rho(x_{n_i}, x_{n_j}) \right\}_{i,j = 1, 2, ..., n} = M
\end{equation*}

\indent Since $X$ is totally bounded there are only finitely many spheres covering $X$, and so this largest distance between points $M$ is itself finite. Therefore,
\begin{align*}
	\rho(y_1, y_2) &\leq \rho(y_1, x_{n_1}) + \rho(x_{n_1}, x_{n_2}) + \rho(x_{n_2}, y_2) \\
	&= \rho(x_{n_1}, x_{n_2}) + 2 \\
	&= M + 2
\end{align*}

\indent That is, $M + 2$ is some bound between arbitrary points $y_1, y_2 \in X$, and so $X$ is indeed bounded. Hence
\begin{equation*}
	\text{A totally bounded metric space $X \implies X$ is bounded,}
\end{equation*}

as desired.
\end{proof}

%
% Proposition
%
{\bf Proposition: {\em Sequentially compact $\implies$ totally bounded}.} 

\begin{proof} Suppose {\em not}. That is, suppose that there exists some $\epsilon > 0$ so that $X$ is {\em not} covered by finitely many spheres $S_{x,\epsilon}$. \\

\indent Pick any point $x_1 \in X$. Note that under our contradictory assumption we have that $S_{x_1, \epsilon} \subsetneq X$. Since there is a point $x_2$ outside of $S_{x_1, \epsilon}$ take the union
\begin{equation*}
	S_{x_1, \epsilon} \cup S_{x_2,\epsilon} \subsetneq X
\end{equation*}

Similarly, there exists some $x_3 \in X$ so that
\begin{align*}
	\rho(x_1, x_3) &\geq \epsilon \\
	\rho(x_2, x_3) &\geq \epsilon
\end{align*}

with union
\begin{equation*}
	S_{x_1, \epsilon} \cup S_{x_2,\epsilon} \cup S_{x_3, \epsilon} \subsetneq X
\end{equation*}

\indent Continue defining points $x_n$ so that we produce the sequence. However, recall that if $X$ is sequentially compact then all subsequences have convergent subsequences. Therefore our sequence $(x_1, x_2, x_3, ...)$ has some convergent subsequence $(x_{n_k})$ in which the distance
\begin{equation*}
	\rho(x_i, x_j) < \frac{\epsilon}{2}
\end{equation*}

However, we have defined $x_n$ to satisfy
\begin{equation*}
	\rho(x_i, x_j) \geq \epsilon
\end{equation*}

Contradiction! Therefore, is $X$ is totally bounded. That is, 
\begin{equation*}
	\text{If a metric space $X$ sequentially compact then it must be totally bounded,}
\end{equation*}

as desired.
\end{proof}

%
% Example
%
{\bf Example:} Consider now some metric space $X$ and $\mathcal U$ an open cover of $X$ so that
\begin{equation*}
	\mathcal U = \{O_i\}
\end{equation*}

and
\begin{equation*}
	X \subset \bigcup_{i\in I} O_i
\end{equation*}

Consider some open set $O_x$ containing a point $x$ and produce the open sphere $S_{x,\delta}$ so that 
\begin{equation*}
	S_{x,\epsilon} \subset O_x
\end{equation*}

\indent Note that every point $x$ has will have its own $\delta$ so that $S_{x,\delta} \subset O_x$. In general, as we approach the boundaries of $X$ we will find this radius $\delta\to 0$. \\

%
% Example
%
{\bf Example:} Cover the set of naturals $\N$ by the open covers $\left\{\left(n - \frac{1}{n}, n + \frac{1}{n}\right)\right\}$ where $\delta = \frac{1}{n}$. Clearly as $n \to \infty$ we will find $\delta\to 0$. However, since with $\delta = 0$ we have the covers $\{n\}$, which still covers $\N$. \\

%
% Definition
%
{\bf Definition: {\em (Lebesgue Number)}.} Let $\mathcal U$ be some covering of $X$ with $\mathcal U = \{U_i\}$. The value $\epsilon > 0$ is called a \underline{Lebesgue number} of the covering if, for all $\delta < \epsilon$,
\begin{equation*}
	\forall\,x\in X,~\exists\,O\in\mathcal U,~S_{x,\delta} \subset O
\end{equation*}

%
% Example
%
{\bf Example:} Let $X = \R$ under the discrete metric $\rho_d$
\begin{equation*}
	\rho_d(x,y) =
	\begin{cases}
		1 & \text{if $x \neq y$} \\
		0 & \text{else}
	\end{cases}
\end{equation*}

\indent Let $\mathcal U$ be an open covering of $X$. Does $\mathcal U$ have a Lebesgue number? We may be able to see that $\epsilon = 1$ works as a Lebesgue number since for all $\delta < \epsilon = 1$ we find that $\rho(x, y) < \delta < \epsilon = 1 \implies x = y \iff \rho_d(x,y) = 0$, so with our open spheres $S_{x,\delta}$
\begin{equation*}
	S_{x,\delta} = \{x~:~\rho_d(x,y) < \delta\}
\end{equation*}

we see that for this $\epsilon = 1$
\begin{equation*}
	\forall\,x\in X,~\exists\,O\in\mathcal U,~S_{x,\delta} \subset O
\end{equation*}

and so $\epsilon = 1$ is indeed a Lebesgue number for $X = \R$ under the discrete metric. \\

%
% Proposition
% 
{\bf Proposition: {\em If $X$ is a sequentially compact metric space and $\mathcal U$ some open covering of $X$ then $\mathcal U$ has a Lebesgue number}.}
\begin{proof} {\em (Proof later)}
\end{proof}

%
% Borel-Lebesgue Theorem
%
{\bf Borel-Lebesgue Theorem:} Let $X$ be a metric space. The following are equivalent:
\begin{enumerate}
	\item $X$ is compact.
	\item $X$ has the Bolzano-Weierstrass property.
	\item $X$ is sequentially compact.
\end{enumerate}

\begin{proof} We have already proven that $1 \implies 2 \implies 3$. All that remains is to prove that $3 \implies 1$. That is, all we must prove is that if $X$ is sequentially compact then $X$ is compact. \\

\indent Assume that $X$ is sequentially compact. Take an open covering $\mathcal U$ of $X$. We have two tools under our disposal: (a) From an earlier result, $X$ is totally bounded and (b) From our above Theorem, $X$ has a Lebesgue number. \\

\indent In particular, let $\epsilon > 0$ be the Lebesgue number of the covering $\mathcal U$. Take $0 < \delta < \epsilon$. Since $X$ is sequentially compact we have that $X$ is totally bounded from our previous work. Therefore, by total boundedness of $X$, there exists finitely many open covers $\{S_{x_i,\delta}\}^n_{i = 1}$ that cover $X$. \\

\indent Since $\delta < \epsilon$ and $\epsilon$ is the Lebesgue number of $\mathcal U$, each of these open spheres in $X$ must lie within an open set $U_i$ from $\mathcal U$. That is,
\begin{align*}
	S_{x_1,\delta} &\subset O_1 \\
	S_{x_2,\delta} &\subset O_2 \\
	&\vdots \\
	S_{x_n,\delta} &\subset O_n
\end{align*}

Since $S_{x_i,\delta} \subset O_i$ and $X \subset \bigcup^n_{i = 1} S_{x_i,\delta}$ by total boundedness, we conclude that
\begin{equation*}
	X \bigcup^n_{i = 1} O_i \quad O_i \in \mathcal U
\end{equation*}

and so, for any open cover $\mathcal U$ there is a finite subcover $\{O_i\}^n_{i = 1} \subset \mathcal U$. That is, if $X$ is sequentially compact then $X$ is compact, as desired.
\end{proof}

%
% Proposition
% 
{\bf Proposition: {\em A closed subset of a compact space is compact}.} 

\begin{proof} Let $X$ be some compact space and $F$ a closed subset of $X$, $F \subset X$. Take any open covering $\mathcal O = \{O_i\}$ so that
\begin{equation*}
	F \subset \bigcup_{i \in I} O_i
\end{equation*}

\indent Consider the complement $X\setminus F$. Since $F \subset \bigcup_{i \in I} O_i$ we find that our compact space $X$ is covered by the union
\begin{equation*}
	X \subset \left(X\setminus F\right) \cup \{O_i\}_{i\in I}
\end{equation*}

However, $X$ is compact. Thus, we can find finitely many such sets
\begin{equation*}
	X \subset \left(X\setminus F\right) \cup \{O_i\}^n_{i = 1}
\end{equation*}

and since $X\setminus$ obviously does not contain $F$ and $F\subset X$ we see that all points in $F$ must be contained by the finite set $\{O_i\}^n_{i = 1}$ so that
\begin{equation*}
	F \subset X \subset \left(X\setminus F\right) \cup \{O_i\}^n_{i = 1}
\end{equation*}

In particular,
\begin{equation*}
	F \subset X \subset \{O_i\}^n_{i = 1}
\end{equation*}

since $F \cap \left(X\setminus F\right) = \emptyset$. Thus, $F$ is covered by finitely many open sets $\{O_i\}^n_{i = 1}$ and so $F$ must be compact, as desired.
\end{proof}

%
% Proposition
%
{\bf Proposition: {\em A compact subspace of a metric space is closed and bounded}.}

\begin{proof} Let $X$ be a metric space and $K$ a compact subspace so that $K \subset X$. Let $y$ be some point of closure of $K$. We have proven that the function $f ~:~ K \to \R$ given by
\begin{equation*}
	f(x) = \rho(x,y)
\end{equation*}

is a continuous functon on $K$. Computing the infimum on $f$ we see that
\begin{equation*}
	\inf_{x\in K} f(x) = \inf_{x\in K} \rho(x,y) = 0
\end{equation*}

since $y$ is a point of closure of $K$. However, we have shown that a continuous function on a {\em sequentially compact} (and so compact) metric space must assume its infinum. Therefore, 
\begin{equation*}
	\exists\,z\in K,~f(z) = 0
\end{equation*}

but $f(z) = \rho(z,y)$ and
\begin{equation*}
	\rho(z,y) = 0 \iff z = y
\end{equation*}

\indent Therefore, $f$ assumes its infimum at $y \in K$. Similarly, since $f$ is continuous it must achieve its maximum on $K$, say at point $y'$, since $K$ is compact. Thus,
\begin{equation*}
		0 \leq f(x) \leq M \quad M \in \R
\end{equation*}

and so $K$ is indeed bounded. That is, $K$ a compact subspace of metric space $X$ is a closed and bounded set, as desired.
\end{proof}

%
% Corollary
%
{\bf Corollary: {\em All compact subspaces of real numbers are closed and bounded (reverse of the Heine-Borel Theorem)}.}

\begin{proof} This is an immediate consequence of the above proposition for $X = \R$.
\end{proof}

%
% Proposition
%
{\bf Proposition: {\em A continuous image of a compact space is compact}.} That is, if $X$ is a compact space and $f:X\stackrel{\textrm{onto}}{\to}Y$ is a continuous surjective function then $Y$ is compact. That is, . It turns out that an analogous result holds for Linedelof spaces. That is, if $X$ is a Lindelof space and $f:X\stackrel{\text{onto}}{\to} Y$ is a continuous surjective function, then $Y$ Lindelof. \\

\indent For example, if we have some function $f:[2,3]\to(0, 1)$ it must be the case that $f$ is {\em not} continous since $(0, 1)$ has been shown to not be compact.

\begin{proof} Let $X$ be some compact space and $Y$ some continuous mapping $f:X\to Y$. Let $\mathcal O = \{O_i\}$ be some open covering of $Y$ such that
\begin{equation*}
	Y \subset \bigcup_{i\in I} O_i
\end{equation*}

\indent We wish to show that under $f$, $Y$ is covered by only finitely many such $O_i$. However, we already know that the preimage of a an open set under a continuous function is open. That is, if $O_i$ is an open set and $f$ is continuous
\begin{equation*}
	P_i = f^{-1}(O_i)
\end{equation*}

is open. Since $f$ is {\em onto} we have that $f(x) \in O_i \implies x \in X$ and so by the definition of $f^{-1}$ we have
\begin{equation*}
	X \subset \bigcup_{i\in I} f^{-1}(O_i)
\end{equation*}

So these sets $\left\{ f^{-1}(O_i) \right\}_{i\in I}$ cover our compact space $X$, and since $X$ is compact we can select finitely many such open sets and still cover $X$, i.e.
\begin{equation*}
	X \subset \bigcup^n_{i = 1} f^{-1}(O_i)
\end{equation*}

Applying $f$ to each $f^{-1}(O_i)$ yields $f\left(f^{-1}(O_i)\right) = O_i$ since $f$ is onto. Hence,
\begin{align*}
	f(X) &\subset \bigcup_{i\in I} f\left(f^{-1}(O_i)\right) \\
	\iff Y &\subset \bigcup^n_{i = 1} O_i
\end{align*}

and so $Y$ is covered by finitely many such open sets $O_i$. Thus, $Y$ is indeed compact, as desired.
\end{proof}



























































































\end{document}