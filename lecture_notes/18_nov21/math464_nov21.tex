% --------------------------------------------------------------
% This is all preamble stuff that you don't have to worry about.
% Head down to where it says "Start here"
% --------------------------------------------------------------
 
\documentclass[12pt]{article}
 
\usepackage[margin=1in]{geometry} 
\usepackage{bm} % bold in mathmode \bm
\usepackage{amsmath,amsthm,amssymb,mathtools}
\usepackage{dsfont} % for indicator function \mathds 1
\usepackage{tikz,pgf,pgfplots}
\usepackage{enumerate} 
\usepackage[multiple]{footmisc} % for an adjascent footnote
\usepackage{graphicx,float} % figures
\usepackage{centernot} % for \centernot\implies (wrapped in \nimplies)

%% set noindent by default and define indent to be the standard indent length
\newlength\tindent
\setlength{\tindent}{\parindent}
\setlength{\parindent}{0pt}
\renewcommand{\indent}{\hspace*{\tindent}}

%% some math macros
\newcommand{\norm}[1]{\left\lVert#1\right\rVert} % vector norm
\newcommand*{\vv}[1]{\vec{\mkern0mu#1}} % \vec with arrow on top
\renewcommand{\Re}{\mathfrak {Re}}
\renewcommand{\Im}{\mathfrak {Im}}
\newcommand{\R}{\mathbb R}
\newcommand{\N}{\mathbb N}
\newcommand{\Z}{\mathbb Z}
\renewcommand{\P}{\mathbb P}
\newcommand{\Q}{\mathbb Q}
\newcommand{\E}{\mathbb E}
\newcommand{\F}{\mathbb F}
\newcommand{\C}{\mathbb C}
\newcommand{\X}{\mathbb X}
\newcommand{\powerset}{\mathcal P}
\renewcommand{\L}{\mathcal L}
\newcommand{\var}{\mathrm{Var}}
\newcommand{\Var}{\mathrm{Var}}
\newcommand{\cov}{\mathrm{Cov}}
\newcommand{\Cov}{\mathrm{Cov}}
\newcommand{\gm}{\mathrm{gm~}}
\newcommand{\am}{\mathrm{am~}}
\newcommand{\trace}{\mathrm{trace~}}
\newcommand{\Trace}{\mathrm{Trace~}}
\newcommand{\rank}{\mathrm{rank~}}
\newcommand{\Rank}{\mathrm{Rank~}}
\newcommand{\Span}{\mathrm{Span~}}
\newcommand{\card}{\mathrm{card}}
\newcommand{\Card}{\mathrm{Card}}
\newcommand{\limplies}{~\Longleftarrow ~} % leftwards implies, for some reason requires spacing to mirror the formatting of \implies (and therefore \rimplies below)
\newcommand{\rimplies}{\implies} % rightwards implies (for consistency)
\newcommand{\nimplies}{\centernot\implies} % rightwards implies with line struck through
\newcommand{\indist}{\,{\buildrel \mathcal D \over \sim}\,}
\newcommand\defeq{\mathrel{\stackrel{\makebox[0pt]{\mbox{\normalfont\tiny 
	def}}}{=}}} % equal sign with def above
\newcommand{\cl}{\mathrm{cl}} % closure of a set cl(X)
\newcommand{\Cl}{\mathrm{Cl}} % closure of a set Cl(X)
\newcommand{\closure}{\mathrm{closure}} % closure of a set closure(X)
\newcommand{\Closure}{\mathrm{Closure}} % closure of a set Closure(X)

\begin{document}
 
% --------------------------------------------------------------
%                         Start here
% --------------------------------------------------------------
 
\title{Real Analysis\\Lecture Notes}
\author{Metric Spaces}
\date{November 21 2016 \\ Last update: \today{}}
\maketitle

\section{Compactness}

Recall that last time we introduce the notion of a {\em Lebesgue number}: \\

%
% Definition
%
{\bf Definition: {\em (Lebesgue number)}.} Let $\mathcal U$ be some open covering of $X$, $\mathcal U = \{U_i\}_{i \in I}$. We say that $\epsilon > 0$ is a \underline{Lebesgue number} of $\mathcal U$ if
\begin{equation*}
	\forall\,\delta < \epsilon,~\forall\,x\in X,~\exists\,O\in\mathcal U,~S_{x,\delta} \subset O
\end{equation*}

%
% Proposition
%
{\bf Proposition:} If $X$ is a sequentially compact space and $\mathcal U$ is an open covering of $X$, then $\mathcal U$ has a {\em Lebesgue number}.

\begin{proof} Suppose $X \in \mathcal U$. That is, $X$ is one of the open sets $O \in \{U_i\}_{i\in I} = \mathcal U$. Then we see that $\delta = 2$ satisfies our definition of the Lebesgue number of $X$ since for $\delta = 2$ we have
\begin{equation*}
	\forall\,x\in X,~\exists\,O\in\mathcal U,~S_{x,2} \subset O
\end{equation*}

and $S_{x,2} \subset O = X \in \mathcal U$ with
\begin{equation*}
	S_{x,2} = \{y\in X~:~ \rho(x,y) < 2\}
\end{equation*}

So if $X \in \mathcal U$, $X$ trivially has a Lebesgue number. \\

Assume now that $X \notin \mathcal U$ so that $X$ may only be the nontrivial union of open sets
\begin{equation*}
	X \subset \bigcup_i U_i
\end{equation*}

for each $U_i \in \mathcal U$. Define the function $\phi:X\to\R$ such that
\begin{equation*}
	\phi(r) = \sup \{ r ~:~ \exists\,O\in\mathcal U,~ S_{x,r} \subset O \}
\end{equation*}

\indent That is, $\phi$ finds the largest radius such that the open spheres $S_{x,r}$ are still enclosed within open sets of the covering $\mathcal U$. We claim that
\begin{equation*}
	0 < \phi(r) < \infty
\end{equation*}

Since $X$ is sequentially compact by assumption we have that 
\begin{align*}
	\text{$X$ is sequentially compact } &\implies \text{ $X$ is totally bounded} \\
	&\implies \text{ $X$ is bounded.}
\end{align*}

Let $M \in \R$, $M < \infty$ be our bounding constant so that $\forall\,x\in X,~-M < x < M$. Thus,
\begin{equation*}
	\forall\,y\in X,~\rho(x,y) < M
\end{equation*}

If $\phi(x) = \infty$ then there is some finite radius $r \in R$, $r < \phi(x)$, with $r > M$ such that
\begin{equation*}
	S_{x,r} \subset O \in \mathcal U
\end{equation*}

but
\begin{equation*}
	X \subset S_{x,r} \subset O 
\end{equation*}

and
\begin{equation*}
	O \subset X
\end{equation*}

by assumption that $X$ is not composed of a single set $O \in \mathcal U$. Thus,
\begin{equation*}
	X \subset O \subset X
\end{equation*}

so
\begin{equation*}
	X = O
\end{equation*}

which is solved by case 1. \\

\indent Thus, in either case $X \in \mathcal U$ or $X \notin \mathcal U$ we have that $X$ has a Lebesgue number $\delta$. That is, if $X$ is a sequentially compact set then a covering $\mathcal U$ of $X$ has a Lebesgue number, as desired. \\

We now consider the general case without setting $\delta = 2$.

%
% Claim
%
{\bf Claim:} With the above definition of $\phi(x) = \sup \{ r ~:~ \exists\,O\in\mathcal U,~ S_{x,r} \subset O \}$ we claim that
\begin{equation*}
	\forall\,x,y\in X,~\phi(y) \geq \phi(x) - \rho(x,y)
\end{equation*}

\begin{proof} If $\phi(x) \leq \rho(x,y)$ then our claim trivially holds since $\phi(x) - \rho(x,y) \leq 0$ and $\phi(y) > 0$ by definition. \\

\indent So, we need only consider the case $\phi(x) > \rho(x,y)$. That is, suppose the distance between any two points in $x,y \in X$ is {\em less than} the maximal radius that the open sphere $S_{x,r}$ may be to still remain fully closed by at least one open sets $O \in \mathcal U$. \\

Consider the difference $r - \rho(x,y)$. Since $\phi(y)$ is the greatest such radius we must have
\begin{equation*}
	r - \rho(x,y) \leq \phi(y)
\end{equation*}

since $\rho(x,y) \geq 0$ by assumption. Taking the sup over all such $r$ yields our result, for $\phi(y) = \sup_{x\in X} r$
\begin{equation*}
	\phi(y) \geq \phi(x) - \rho(x,y)
\end{equation*}

as desired.
\end{proof}

Now, using this result note that if $\phi(y) \geq \phi(x) - \rho(x,y)$ then
\begin{align*}
	\phi(y) &\geq \phi(x) - \rho(x,y) \\
\iff \phi(x) &\geq \phi(y) - \rho(x,y)
\end{align*}

by symmetry since $x$ and $y$ are arbitrary points in $X$. Noting that the distance $\rho(x,y)$ is identical in both inequalities we may state that
\begin{equation*}
	|\phi(x) - \phi(y)| \leq \rho(x,y)
\end{equation*}

So, for $\rho(x,y) < \delta = \epsilon$ we get that
\begin{equation*}
	\rho(x,y) < \delta \implies |\phi(x) - \phi(y)| < \epsilon
\end{equation*}

and so $\phi$ is a continuous function. That is, if our inequality is indeed true we get that $\phi$ is a continuous function. However, since $X$ is sequentially compact by assumption, we have that all function $f:X\to\R$ {\em must} be continuous, and so $\phi$ {\em must} be a continuous function. \\

\indent Since $\phi$ must be continuous we know that it must achieve its infimum. So, there exists some $x$ for which $\inf_{x\in X} \phi(x)$ is attained. Let this infimum $\inf_{x\in X} \phi(x)= \epsilon$. We claim that this $\epsilon = \inf_{x\in X}$ is a Lebesgue number for our cover $\mathcal U$. \\

\indent To see this take $\delta$ so that $0 < \delta < \epsilon$. The small value of $\phi$ must be $\epsilon$ by definition of $\epsilon = \inf_{x\in X} \phi(x)$ (since $\phi$ achieves its infimum on $X$). Therefore,
\begin{equation*}
	\forall\,x\in X,~\delta < \phi(x)
\end{equation*}

since $\delta < \epsilon = \inf_{x\in X} \phi(x) <  \phi(x)$. Hence,
\begin{equation*}
	\delta < \sup \{r ~:~ S_{x,r} \subset O \in \mathcal U\}
\end{equation*}

or more completely
\begin{equation*}
	\forall\,\delta < \epsilon,~\forall\,x\in X,~\exists\,O\in\mathcal U,~S_{x,\delta} \subset O
\end{equation*}

which is precisely the definition for $\epsilon$ satisfying the definition of a Lebesgue number for $\mathcal U$. That is, since $X$ is compact there will always be such $\delta < \epsilon = \inf_{x \in X} \phi(x)$ to act as a Lebesgue number since $\phi(x)$ is continuous and thus it must achieve such an infimum. Thus, for a compact set $X$, $\epsilon = \inf_{x \in X} \phi(x)$ works as our general Lebesgue number for our cover $\mathcal U$ of $X$, as desired.
\end{proof}

{\bf A good final question may be to prove such a result under the restriction that $X$ is a metric space.} \\

%
% Definition
%
{\bf Definition: {\em (Compactification)}.} Let $X$ be some topological space. We say that a \underline{compactification} of $X$ is a compact space $K$ such that $X$ is a {\em dense subset} of $K$. \\

Recall for a moment the definition of a {\em dense subset} $X$ of $K$:
\begin{equation*}
	\forall\, O\neq \emptyset, O \text{ open in K},~O \cap X \neq \emptyset
\end{equation*}

\indent That is, $X$ is a dense subset of $K$ if all open subsets of $K$ form nonempty intersections with $X$. Equivalently, $X$ is dense in $K$ if
\begin{equation*}
	\overline{X} = K
\end{equation*}

\indent For example, the set of rationals $\Q$ is dense in $\R$ since any open set in $\R$ must contain at least one $q \in \Q$. However, $\N$ is not dense in $\R$ since we may construct nonempty open sets that do not contain a single natural number $n \in \N$, say $(2,3) \subset \R$. \\

%
% Example
%
{\bf Example: {\em (Example of a compactification)}.} Consider the real interval $(0, 1)$. We have shown that such an interval is not compact since we were able to construct a covering with no finite subcover. However, we can construct the set $[0,1]$ which will be the {\em two-point compactification} of $(0,1)$ since the closure of $(0,1)$ is exactly $[0,1]$, and so $(0,1)$ is dense in $[0,1]$.
















































































\end{document}