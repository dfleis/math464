% --------------------------------------------------------------
% This is all preamble stuff that you don't have to worry about.
% Head down to where it says "Start here"
% --------------------------------------------------------------
 
\documentclass[12pt]{article}
 
\usepackage[margin=1in]{geometry} 
\usepackage{bm} % bold in mathmode \bm
\usepackage{amsmath,amsthm,amssymb,mathtools}
\usepackage{dsfont} % for indicator function \mathds 1
\usepackage{tikz,pgf,pgfplots}
\usepackage{enumerate} 
\usepackage[multiple]{footmisc} % for an adjascent footnote
\usepackage{graphicx,float} % figures
\usepackage{centernot} % for \centernot\implies (wrapped in \nimplies)

%% set noindent by default and define indent to be the standard indent length
\newlength\tindent
\setlength{\tindent}{\parindent}
\setlength{\parindent}{0pt}
\renewcommand{\indent}{\hspace*{\tindent}}

%% some math macros
\newcommand{\norm}[1]{\left\lVert#1\right\rVert} % vector norm
\newcommand*{\vv}[1]{\vec{\mkern0mu#1}} % \vec with arrow on top
\renewcommand{\Re}{\mathfrak {Re}}
\renewcommand{\Im}{\mathfrak {Im}}
\newcommand{\R}{\mathbb R}
\newcommand{\N}{\mathbb N}
\newcommand{\Z}{\mathbb Z}
\renewcommand{\P}{\mathbb P}
\newcommand{\Q}{\mathbb Q}
\newcommand{\E}{\mathbb E}
\newcommand{\F}{\mathbb F}
\newcommand{\C}{\mathbb C}
\newcommand{\X}{\mathbb X}
\newcommand{\powerset}{\mathcal P}
\renewcommand{\L}{\mathcal L}
\newcommand{\var}{\mathrm{Var}}
\newcommand{\Var}{\mathrm{Var}}
\newcommand{\cov}{\mathrm{Cov}}
\newcommand{\Cov}{\mathrm{Cov}}
\newcommand{\gm}{\mathrm{gm~}}
\newcommand{\am}{\mathrm{am~}}
\newcommand{\trace}{\mathrm{trace~}}
\newcommand{\Trace}{\mathrm{Trace~}}
\newcommand{\rank}{\mathrm{rank~}}
\newcommand{\Rank}{\mathrm{Rank~}}
\newcommand{\Span}{\mathrm{Span~}}
\newcommand{\limplies}{~\Longleftarrow ~} % leftwards implies, for some reason requires spacing to mirror the formatting of \implies (and therefore \rimplies below)
\newcommand{\rimplies}{\implies} % rightwards implies (for consistency)
\newcommand{\nimplies}{\centernot\implies} % rightwards implies with line struck through
\newcommand{\indist}{\,{\buildrel \mathcal D \over \sim}\,}
\newcommand\defeq{\mathrel{\stackrel{\makebox[0pt]{\mbox{\normalfont\tiny 
	def}}}{=}}} % equal sign with def above

\begin{document}
 
% --------------------------------------------------------------
%                         Start here
% --------------------------------------------------------------
 
\title{Real Analysis\\Lecture Notes}
\author{Set Theory}
\date{September 7 2016 \\ Last update: \today{}}
\maketitle

\section{Ordered Sets}

\indent We begin with introducing the notion of a \underline{partially ordered set}. Suppose $S$ is partially ordered. Given some binary relation $R$ defined over elements $a, b \in S$, the set $S$ must have a partial order that satisfies the following
\begin{enumerate}[(i)]
	\item $aRa \quad \forall\,a\in S$ (reflexivity)
	\item $aRb \land bRa \implies a = b$ (symmetry)
	\item $aRb \land bRc \implies aRb$ (transitivity)
\end{enumerate}

\underline{Example}: Consider the finite set $X = \{1,2,3,4\}$, and consider its power set $\powerset(X) = S = \{ \text{all subsets of $X$} \}$. Then, for elements $A, B \in S$, does the relation `$\subseteq$' (where $\subseteq$ is defined in the typical way) satisfy a partial order over $S$? \\

\indent To verify that $\subseteq$ is indeed a partial order we simply go through our three criteria defined above. For this particular example we can confirm our criteria via explicitly outlining the elements of our set $S$ (this may not be feasible in sufficiently large sets, or possible in infinite sets). So, we find that\footnote{Note that if a finite set $X$ has $n$ elements then $\powerset(X)$ will contain $2^n$ elements.}
\begin{align*}
	S &= \powerset(X) \\
	&= \Big\{ \emptyset, \{1\}, \{2\}, \{3\}, \{4\}, \\
	&\hphantom{{}={--}} \{1,2\}, \{1,3\}, \{1,4\}, \{2,3\}, \{2,4\}, \{3,4\}, \\
	&\hphantom{{}={--}} \{1,2,3\}, \{1,2,4\}, \{1,3,4\}, \{2,3,4\}, \{1,2,3,4\} \Big\}
\end{align*}

Now, going through our criteria:
\begin{enumerate}[(i)]
	\item $A \stackrel{?}{\subseteq} A \quad \forall A \in S$. Using the definition of $\subseteq$, we should find that the reflexivity criteria is easily satisfied. 
	\item $A \subseteq B \land B \subseteq A \stackrel{?}{\implies} A = B$. Once again, our finite set permits us to manually verify this condition. We see that the only two elements $A, B$ which satisfy the symmetry condition are when $A$ and $B$ are chosen such that $A = B$.
	\item $A \subseteq B \land B \subseteq C \stackrel{?}{\implies} A \subseteq C$. Perhaps this is a bit less trivial to verify, but once again we should quickly see that this is indeed the case for all elements $A, B, C \in S$, and so the transitivity of our relation $\subseteq$ holds.
\end{enumerate}

\indent 	Thus, $\subseteq$ is indeed a partial order, and so the set $S = \mathcal P(A)$ under $\subseteq$ forms a partially ordered set. It turns out that we call this relation $\subseteq$ the {\em containment relation}. \\

\indent However, using the above example, despite being a partially ordered set we have elements who cannot be so easily compared/ordered under $\subseteq$. Namely,
\begin{align*}
	\{1, 2\} &\nsubseteq \{1, 3\} \\
	\{2, 3\} &\nsubseteq \{1, 3, 4\}
\end{align*}

\indent It is not immediately clear how we should deal with these cases under a partial order. For this reason we choose to introduce the notion of a \underline{total order} and \underline{totally ordered sets}. \\

\indent Consider the closed interval $S = [0, 1]$ and the relation $a \leq b$ (as typically defined). If we were willing to do so, we could verify that $\leq$ is indeed a partial order on $S$. However, it is also true that {\em all} $a, b \in S$ satisfies either
\begin{equation*}
	a \leq b \quad \text{or} \quad b \leq a \quad \forall\,a,b\in S
\end{equation*}

\indent Hence, if $S$ is a \underline{totally ordered set}, then for some binary relation $R$ over a set $S$, we require the following four criteria:
\begin{enumerate}[(i)]
	\item $\forall\,a\in S, ~ aRa$ (reflexivity)
	\item $aRb \land bRa \implies a = b$ (symmetry)
	\item $aRb \land bRc \implies aRb$ (transitivity)
	\item $ \forall\,a,b\in S, ~ aRb \text{ or } bRa$ (comparability)
\end{enumerate}

\indent Consider the finite set of integers $\{1,2,3,4\}$ and the set of nonnegative real numbers $[0, \infty)$ under the usual order.\footnote{Typically, the ``usual order'' will mean $\leq$ if the elements are numbers or $\subseteq$ if the elements are sets.} We may verify that both these sets are indeed totally ordered, and yet there turns out to be (at least one) material distinction between the two. \\

\indent We pose the question: Does any nonempty subset of $S$ have a ``smallest''\footnote{This means what you think it means: For some total order $\leq$, we find $\forall\,s \in S,~\exists\,t\in S,~ t\leq s$} element? (ed. For some reason I prefer calling it a {\em least element}) Clearly this is trivially true for $S = \{1,2,3,4\}$: If $1 \in S' \subset S$ then 1 is our least element. If $1 \notin S'$ then, if $2 \in S'$ then 2 is our least element, and so on... Consider now our second set $S = [0, \infty)$. If we select the subset $(0, 1) \subseteq [0, \infty)$ we may note that there is no $x \in (0, 1)$ such that for all $y \in (0, 1)$ we find $x \leq y$ since, as an open interval, we may descend arbitrarily close to the left endpoint. \\

\indent For the above reason we introduce the notion of a \underline{well ordered set}. We say that a \underline{well ordered set} is a totally ordered set such that all nonempty subsets has a least element. From this definition it should be clear that the set of naturals $\N$ is well ordered but the set of real numbers $\R$ is not, since we may construct open subsets without a least element (as illustrated above). \\

\indent Under this definition we note that the set $S = \{..., -3, -2, -1, 0, 1\}$ is not well ordered since, although we may find a greatest element, if we select the subset $\{..., -3, -2\}$ we see that there is no least element. \\

\indent Now, suppose $T$ is totally ordered by the total order $\leq$. Let $T_1 \subseteq T$, $T_1 \neq \emptyset$. If $T_1$ inherits a total order from its totally ordered parent $T$ then it's easy to see that it remains totally ordered itself:
\begin{proof} For $a,b,c \in T_1$ we have that since $T_1 \subseteq T$, $a,b,c \in T$. Since $T$ is totally ordered we have that $\leq$ satisfies the four criteria of total ordered above with respect to $a,b,c$. However, since $a,b,c$ were chosen to be arbitrary elements of $T_1$, we may conclude that $T_1$ is indeed totally ordered, as desired.
\end{proof}

\indent Suppose $T$ is now a well ordered set. It is perhaps less clear whether $T_1 \subseteq T$ is well ordered. However, it turns out that the subset $T_1$ does indeed remain well ordered:
\begin{proof} Suppose $T$ is well ordered and let $T_1 \subseteq T$. If $T$ is well ordered then $T$ is totally ordered and all nonempty subsets of $T$ contain a least element. Since we have shown that $T_1 \subseteq T$ must be totally ordered, it is sufficient to show that all nonempty subsets of $T_1$ contain a least element. However, all subsets of $T_1$ must be subsets of $T$ since the elements of $T_1$ are wholly contained by $T$. Since the nonempty subsets of $T$ have a least element, we may conclude that the nonempty subsets of $T_1$ must have a least element, as desired.
\end{proof}

\indent It turns out that being well ordered on $\N$ is identical to the principle of mathematical induction.\footnote{This was given as an assignment problem in class.} That is, if a given set is well ordered then we may use mathematical induction to prove some statement is true for all elements of our set. We quickly summarize mathematical induction: If we have given statements indexed by $n \in \N$, say $P(1), P(2), ..., P(n),...$ we know that all statements $P(n), n \in \N$ hold if
\begin{enumerate}
	\item $P(1)$ is true.
	\item $P(k) \implies P(k + 1)$.
\end{enumerate}


\section{Sequences}

\indent A \underline{finite sequence} from $X$ is a function from a set of positive integers $\{1,2,...,n\}$ into $X$. We may denote this by
\begin{align*}
	f\{1,2,..., n\} \longrightarrow X \quad \text{or} \\
	\{x_1, x_2, ..., x_n\} \quad n \in \N
\end{align*}

\indent A \underline{countably infinite sequence} from $X$ is a function $f:\N\to X$ where $\N$ has the natural order. A \underline{countably infinite set} is a set which is in the range of a countably infinite sequence. \\

\underline{Example}: Is the set $S = \{-1, -2, -3, ... \}$ countably infinite? Yes! Define $f:\N \to S$ such that
\begin{align*}
	f(1) &= -1 \\
	f(2) &= -2 \\
	f(3) &= -3  \\
	&\vdots \\
	f(n) &= -n \\
	&\vdots
\end{align*}

\indent We should note that this function $f$ is surjective (onto) since all elements of $S$ are mapped to by $f$ from some element of $\N$, and so we have found a function $f:\N \to S$ to satisfy the definition of a countably infinite set. \\

\underline{Example:} Consider the set $S = \{0,2,4,8,...\}$. We may define $f:\N\to S$ by $f(n) = 2n - 2$ to show that $S$ is indeed countably infinite. \\

\indent So, we have defined a countably infinite set $S$ to be a set that may be mapped from by $\N$ by $f:\N\to S$. However, it is sometimes inconvenient to construct a mapping from $\N$ to $S$. For this reason we introduce the following result: If we map a set $S$ to a second set $T$ by $g:S\to T$ we immediately see that the composition $(g\circ f)(n) = g(f(n))$, $g\circ f:\N \to T$, will map $\N$ onto $T$. Therefore, if $S$ is countable and if there exists some map from $S$ onto $T$, then $T$ must also be countable. Hence, we may prove that some set $T$ is countable by showing that there exists some surjective mapping from $S$ onto $T$.

\subsection{Subsequences}

\indent For some sequence from the set $X$, say $\{x_1, x_2, x_3 ...\}$, we define a \underline{subsequence of a sequence} to be the the sequence $\{x_{n_1}, x_{n_2}, x_{n_3},...\}$ with indices $n_1, n_2, n_3, ... \in \N$. Alternatively, if 
\begin{equation*}
	f\{1,2,...,\}
\end{equation*}

is some sequence, then the set
\begin{equation*}
	h\{1,2,...\}
\end{equation*}

is a subsequence of $f\{1,2,...\}$ if there exists a monotone function $g:\N\to\N$ such that
\begin{equation*}
	h = g \circ f
\end{equation*}

That is, our subsequence $\{x_{n_1}, x_{n_2},...\}$ can be given by
\begin{align*}
	x_{n_1} = f(n_1) &= g(f(n_1)) = h(1)  \\
	x_{n_2} = f(n_2) &= g(f(n_2)) = h(2) \\
	&\vdots
\end{align*}

\section{Unions, Intersection, and Complementation}

It will be useful to recall DeMorgan's Laws:
\begin{enumerate}[(1)]
	\item $(A \cap B)^c = A^c \cup B^c$
	\item $(A \cup B)^c = A^c \cap B^c$
\end{enumerate}

(ed. Essentially everything else was omitted from this section of the text?)

\section{Algebras of Sets}

Let $X$ be a set. An \underline{algebra} $\mathcal A$ of subsets of $X$ is a collection of sets closed under intersection, union, set difference,\footnote{For sets $X$ and $Y$ we say that ``set subtraction is given by $X \setminus Y = X \cap Y^c$.} and complementation as follows: For $A, B \in \mathcal A$, then
\begin{enumerate}[(a)]
	\item $A \cap B \in \mathcal A$
	\item $A \cup B \in \mathcal A$
	\item $B \setminus A \in \mathcal A$
	\item $A^c \in \mathcal A$
\end{enumerate}

\indent Strictly speaking, we only need $A \cup B \in \mathcal A$ and $A^c \in \mathcal A$ in order to derive the remaining two properties. For example, $A \cap B = (A^c \cup B^c)^c$ and so since $A, B \in \mathcal A$, and since $\mathcal A$ is closed under complementation, we find $A^c, B^c \in \mathcal A$. Since $\mathcal A$ is closed under unions we find $A^c \cup B^c \in \mathcal A$. Finally once again realizing $\mathcal A$ is closed under complementation we find $(A^c \cup B^c)^c \in \mathcal A$. We can perform similar manipulations to show that the explicit statement of $B \setminus A = B \cap A^c \in \mathcal A$ is not necessary. \\

\underline{Example}: Let $X = \N$ and consider the collection $\mathcal A = \{\emptyset, \N\}$. We can verify that our closure criteria over our two elements to conclude that $\mathcal A$ is indeed an algebra. \\

\underline{Example}: Let $X = \N$ and $\mathcal B = \{ \text{all subsets of $\N$} \}$. This collection can too be shown as a valid algebra. \\

\underline{Example}: {\em (Boolean algebra on $\N$)} Take $X = \N$ and \newline $\mathcal C = \{ \text{all subsets of $X$ which are either finite or have finite complements (cofinite)} \}$.\footnote{An example of a set which is neither finite nor cofinite is the set of even naturals over the set of naturals. Its complement is the set of odd naturals which is clearly infinite.} We start with determining what type of elements belong to $\mathcal C$. Listing some examples we find:
\begin{align*}
	\{2,3\} &\in \mathcal C \quad \text{(finite)} \\
	\N \setminus \{5\} &\in \mathcal C \quad \text{(cofinite)} \\
	\N &\in \mathcal C \quad \text{(cofinite)} \\
	\emptyset &\in \mathcal C \quad \text{(finite)}
\end{align*}

\indent From these examples it should be immediately obvious that $\mathcal C$ is indeed closed under complementation.\footnote{I would like to more rigorously prove this, but this is essentially as was stated in class.} \\

\indent What about intersection? If sets $A$ and $B$ are finite then, by definition, $A, B \in \mathcal C$ and clearly $A\cap B$ must be finite, so $A \cap B \in \mathcal C$. Similarly, for cofinite $B$, if $A$ is finite then $A\cap B$ must be finite, and so $A \cap B \in \mathcal C$. Finally, if both $A$ and $B$ are cofinite we must consider $A\cap B$. Note
\begin{align*}
	A\cap B &= \left[ \left( A \cap B \right)^c \right]^c \\
		&= \left[ A^c \cup B^c \right]^c
\end{align*}

but since $A$ and $B$ are cofinite, $A^c$ and $B^c$ must be finite, so $A^c \cup B^c$ must be finite, and because
\begin{equation*}
	\left( A \cap B \right)^c = \left( \left[ A^c \cup B^c \right]^c \right)^c = A^c \cup B^c
\end{equation*}

is finite, we may conclude that, by definition, $A \cap B$ is cofinite. \\

\indent What about unions? If $A, B \in \mathcal C$ are both finite, then clearly $A \cup B$ is finite and so the union must be finite: $A \cup B \in \mathcal C$. If $A$ is finite and $B$ is cofinite, then 
\begin{align*}
	B \subseteq& A \cup B \\
	\implies B^c \supseteq&  (A \cup B)^c \quad \text{(complementation reverses containment\footnotemark)}
\end{align*}
\footnotetext{Draw a diagram if you don't see this.}

\indent  Since $B$ is cofinite, $B^c$ must be finite. Thus, from $(A\cup B)^c \subseteq B^c$ we have that the complement of $A\cup B$ is finite, and so $A\cup B$ is cofinite. Hence, $A\cup B \in \mathcal C$. Finally, if both $A$ and $B$ are both cofinite then we should note that
\begin{align*}
	A\cup B &= \left[ \left( A \cup B \right)^c \right]^c \\
		&= \left[ A^c \cap B^c \right]^c \\
		\implies (A \cup B)^c &= A^c \cap B^c
\end{align*}

Since $A$ and $B$ are cofinite, $A^c$ and $B^c$ must be finite, and clearly the intersection $A^c \cap B^c$ must also be finite. Therefore, by definition, the union $A \cup B \in \mathcal C$, as desired. \\
s
\indent Are all subsets of $\N$ finite or cofinite? No! As a counterexample, consider $A = \{\text{all primes}\}$. We know that this is an infinite set with complement $A^c = \{\text{all commposites}\}$, which is itself infinite set.
\end{document}
















