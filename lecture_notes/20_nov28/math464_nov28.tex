% --------------------------------------------------------------
% This is all preamble stuff that you don't have to worry about.
% Head down to where it says "Start here"
% --------------------------------------------------------------
 
\documentclass[12pt]{article}
 
\usepackage[margin=1in]{geometry} 
\usepackage{bm} % bold in mathmode \bm
\usepackage{amsmath,amsthm,amssymb,mathtools}
\usepackage{dsfont} % for indicator function \mathds 1
\usepackage{tikz,pgf,pgfplots}
\usepackage{enumerate} 
\usepackage[multiple]{footmisc} % for an adjascent footnote
\usepackage{graphicx,float} % figures
\usepackage{centernot} % for \centernot\implies (wrapped in \nimplies)

%% set noindent by default and define indent to be the standard indent length
\newlength\tindent
\setlength{\tindent}{\parindent}
\setlength{\parindent}{0pt}
\renewcommand{\indent}{\hspace*{\tindent}}

%% some math macros
\newcommand{\norm}[1]{\left\lVert#1\right\rVert} % vector norm
\newcommand*{\vv}[1]{\vec{\mkern0mu#1}} % \vec with arrow on top
\renewcommand{\Re}{\mathfrak {Re}}
\renewcommand{\Im}{\mathfrak {Im}}
\newcommand{\R}{\mathbb R}
\newcommand{\N}{\mathbb N}
\newcommand{\Z}{\mathbb Z}
\renewcommand{\P}{\mathbb P}
\newcommand{\Q}{\mathbb Q}
\newcommand{\E}{\mathbb E}
\newcommand{\F}{\mathbb F}
\newcommand{\C}{\mathbb C}
\newcommand{\X}{\mathbb X}
\newcommand{\powerset}{\mathcal P}
\renewcommand{\L}{\mathcal L}
\newcommand{\var}{\mathrm{Var}}
\newcommand{\Var}{\mathrm{Var}}
\newcommand{\cov}{\mathrm{Cov}}
\newcommand{\Cov}{\mathrm{Cov}}
\newcommand{\gm}{\mathrm{gm~}}
\newcommand{\am}{\mathrm{am~}}
\newcommand{\trace}{\mathrm{trace~}}
\newcommand{\Trace}{\mathrm{Trace~}}
\newcommand{\rank}{\mathrm{rank~}}
\newcommand{\Rank}{\mathrm{Rank~}}
\newcommand{\Span}{\mathrm{Span~}}
\newcommand{\card}{\mathrm{card}}
\newcommand{\Card}{\mathrm{Card}}
\newcommand{\limplies}{~\Longleftarrow ~} % leftwards implies, for some reason requires spacing to mirror the formatting of \implies (and therefore \rimplies below)
\newcommand{\rimplies}{\implies} % rightwards implies (for consistency)
\newcommand{\nimplies}{\centernot\implies} % rightwards implies with line struck through
\newcommand{\indist}{\,{\buildrel \mathcal D \over \sim}\,}
\newcommand\defeq{\mathrel{\stackrel{\makebox[0pt]{\mbox{\normalfont\tiny 
	def}}}{=}}} % equal sign with def above
\newcommand{\cl}{\mathrm{cl}} % closure of a set cl(X)
\newcommand{\Cl}{\mathrm{Cl}} % closure of a set Cl(X)
\newcommand{\closure}{\mathrm{closure}} % closure of a set closure(X)
\newcommand{\Closure}{\mathrm{Closure}} % closure of a set Closure(X)

\begin{document}
 
% --------------------------------------------------------------
%                         Start here
% --------------------------------------------------------------
 
\title{Real Analysis\\Lecture Notes}
\author{Metric Spaces}
\date{November 28 2016 \\ Last update: \today{}}
\maketitle

\section{Baire Categories}

%
% Baire Category Theory
%
{\bf Baire Category Theorem:} Let $X$ be a complete metric space and take a countable family of dense open sets $\{O_i\}^\infty_{i = 1}$ from $X$. The intersection
\begin{equation*}
	\bigcap^\infty_{i = 1} O_i
\end{equation*}

is dense. 

\begin{proof} Take a nonempty open set $U$ of $X$. Since we claim that the countable intersection $\bigcap^\infty_{i = 1} O_i$ is dense in $X$ we wish to prove that
\begin{equation*}
	U \cap \left( \bigcap^\infty_{i = 1} O_i \right) \neq \emptyset
\end{equation*}

\indent It turns out that this nonemptyness will emerge as a consequence of completeness. We rewrite this intersection as
\begin{equation*}
	U \cap \left( \bigcap^\infty_{i = 1} O_i \right) = \bigcap^\infty_{i = 1} \left( U \cap O_i \right)
\end{equation*}

and the intersection of each $\left( U \cap O_i \right) \neq \emptyset$ since $O_i$ is each $O_i$ is dense in $X$ so that $\overline{O_i} = X$. \\

\indent Take $x_1 \in U \cap O_1$. Since $U$ and each $O_i$ is open we have that each $\left( U \cap O_i \right)$ is open and so we can form the open spheres
\begin{equation*}
	S_{x_1, r_1} \equiv S_1 \subset U \cap O_1
\end{equation*}

\indent Thus, we remain within our open subset $U$ by remaining within the open sphere $S_1$ since $S_1 \subset U \cap O_1$ and so $S_1 \subset U$. \\

\indent Now, pick $O_2$ in our intersection. We have that $O_2$ is dense and open in $X$. Therefore, much like $O_1$,
\begin{equation*}
	O_2 \cap S_1 \neq \emptyset
\end{equation*}

and $O_2 \cap S_1$ is open since both sets are open. Take $x_2 \in O_2 \cap S_1$ and note that since this intersection is open we may take some open sphere $S_2$ such that
\begin{equation*}
	S_2 = S_{x_2, r_2} \subset O_2 \cap S_1
\end{equation*}

\indent From these points $x_1$ and $x_2$ we have the distance $\rho(x_1, x_2) < r_1 - r_2$ since $x_1 \in S_1 \equiv S_{x_1, r_1}$ and $x_2 \in S_2 \equiv S_{x_2, r_2}$ (draw a diagram). \\

In addition to this construction of $S_1$ and $S_2$ let us insist that $r_2$ is bound above by
\begin{equation*}
	r_2 < \frac{1}{2}r_1
\end{equation*}

so that our sequence of radii $r_n \to 0$. \\

{\bf Claim}: We claim that the closure of $S_2$ is a subset of $S_1$, $\overline{S_2} \subset S_1$. To show this let $y \in \overline{S}_2 \setminus S_2$ so that $y$ is a bound along the closed boundary of $\overline{S}_2$. We find that
\begin{align*}
	\rho(y,x_2) &= r_2 \\
	\rho(y,x_1) &\leq \rho(y, x_2) + \rho(x_2, x_1) \\
	&< r_2 + (r_1 - r_2) \\
	&= r_1 \\
	\implies \rho(y,x_1) &< r_1
\end{align*}

Therefore,
\begin{align*}
	y &\in \overline{S}_2 \setminus S_2 \\
	\implies y &\in S_1 \\
	\implies \overline{S}_2&\setminus S_2 \subset S_1 \\
	\implies \overline{S}_2 &\subset S_1
\end{align*}

Now, take $O_3 \cap S_2$ and $x_3 \in O_3 \cap S_2$ and take the sphere $S_3$
\begin{equation*}
	S_3 = S_{x_3, r_3}
\end{equation*}

such that
\begin{equation*}
	r_3 < \frac{1}{2} r_2 < \frac{1}{4} r_1
\end{equation*}

Then, once again, we have that $\overline{S}_3 \subset S_2$ by the same argument as above. \\

Repeating this process inductively we get 
\begin{equation*}
	r_n < \frac{1}{2(n - 1)} r_1 \longrightarrow 0 \quad \text{as } n\longrightarrow\infty	
\end{equation*}

\indent Thus, our radii of open spheres $S_n \subset O_n\cap S_{n - 1}$ vanish as $n\to\infty$. By this construction we find our sequence of centers $(x_n) = (x_1, x_2, ...)$ is Cauchy since the radii of the open spheres around these points $r_n\to 0$ and so these points must be getting arbitrarily close together. \\

Therefore, by the assumption of the completeness of $X$ we have that $(x_n)\longrightarrow x\in X$. \\

Now, let $N$ be some fixed natural number sufficiently large so that
\begin{equation*}
	x_n \in S_{N + 1} \quad \text{for } n = N + 1, N + 2
\end{equation*}

and since $(x_n) \to x$ we have that $x$ must also be a limit point of this subsequence $(x_{N + 1}, x_{N + 2},...)$. However, by construction of $x_n \in S_{N + 1}$ we have
\begin{equation*}
	\{ x_{N + 1}, x_{N + 2}, x_{N + 3}, ... \} \subset S_{N + 1}
\end{equation*}

Therefore
\begin{equation*}
	x \in \overline{S}_{N + 1}
\end{equation*}

but
\begin{equation*}
	\overline{S}_{N + 1} \subset S_N \subset O_N
\end{equation*}

hence
\begin{align*}
	x &\in O_N \quad \forall\,N \\
	\implies x &\in \bigcap^N_{i = 1} O_i
\end{align*}

and so 
\begin{align*}
	x &\in U \cap \left( \bigcap^\infty_{i = 1} O_i \right) \\
	\implies U &\cap \left( \bigcap^\infty_{i = 1} O_i \right) \neq \emptyset
\end{align*}

\indent Therefore, since $U$ was an arbitrary open subset of $X$ we have that the countable intersection $\bigcap^\infty_{i = 1} O_i$ is dense in $X$, as desired.
\end{proof}

%
% Lemma
%
{\bf Lemma:} Let $A$, $B$, and $C$ be metric spaces such that $A \subset B \subset C$. Suppose that $A$ is dense in $C$. Then, subspace $A$ is also dense in $B$.

\begin{proof} Let $O \subset B$ be some nonempty open set in $B$. To show that $A$ is dense in $B$ we must show that 
\begin{equation*}
	O \cap A \stackrel{?}{\neq} \emptyset
\end{equation*}

Since $B \subset C$ we have some open set in $C$ such that
\begin{equation*}
	V \cap B = O
\end{equation*}

Since $O \neq \emptyset$ we must have $V \neq \emptyset$ and so
\begin{equation*}
	A \cap V \neq \emptyset
\end{equation*}

Thus
\begin{align*}
	A \cap V &= \left( A \cap V \right) \cap B \\
	&= A \cap \left( V \cap B \right) \\
	&= A \cap O \\
	\implies A \cap O &\neq \emptyset
\end{align*}

and so $A$ is dense in $B$, as desired.
\end{proof}

%
% Claim
%
{\bf Claim:} The set of irrationals $\R\setminus\Q$ has the Baire Category property. \\

\indent Note that we have found a Cauchy sequence $(x_n)$ from $\R\setminus\Q$ that does not converge in $\R\setminus\Q$, in particular
\begin{equation*}
	\left( \frac{\pi}{n} \right) \longrightarrow 0
\end{equation*}

and so $\R\setminus\Q$ is {\em not} complete. Thus, if our claim is true we see that the Baire Category property {\em does not imply} completeness. 

\begin{proof} Suppose $O_1, O_2, ...$ are dense and open in $\R\setminus\Q$. We can find an open set $V_1 \subset \R$ such that
\begin{equation*}
	V_1 \cap \left( \R \setminus \Q \right) = O_1
\end{equation*}

Is $V_1$ dense in $\R$? \\

Suppose $P_1 \neq \emptyset$ is open in $\R$ and $P \cap V_1 = \emptyset$. Then
\begin{align*}
	P &\cap \left( \R\setminus\Q \right) \quad \text{is open in $\R\setminus\Q$, and} \\
	P &\cap \left( \R\setminus\Q \right) \neq \emptyset
\end{align*}

since $\R\setminus\Q$ is dense in $\R$ (and so an intersection of an open set with a dense set is nonempty). \\

Since $O_1$ is dense in $\R\setminus\Q$ by assumption we have (since $P \cap \left( \R\setminus\Q \right) \neq \emptyset$)
\begin{align*}
	\left( P \cap \left( \R\setminus\Q \right) \right) \cap O_1 &\neq \emptyset \\
	\left( P \cap \left( \R\setminus\Q \right)\right) \cap O_1 &\subset V_1 \cap \left( \R\setminus\Q \right)
\end{align*}

Therefore, the countable intersection
\begin{align*}
	\left( \bigcap^\infty_{n = 1} \underbrace{V_n}_{\text{dense and open in $\R$}} \right) \bigcap_{q\in\Q} \underbrace{\left(\R\setminus\{q\}\right)}_{\text{open and dense in $\R$}} &= \bigcap^\infty_{n = 1} V_n \cap \left( \R\setminus\Q \right) \\
	&= \bigcap^\infty_{n = 1} \left( V_n \cap \left(\R\setminus\Q\right)\right) \\
	&= \bigcap^\infty_{n = 1} O_n
\end{align*}

\indent Since each $O_n$ is dense in $\R$ we have by our lemma each $O_n$ is dense in $\R\setminus\Q$. Thus, the Baire Category property is satisfied for $\R\setminus\Q$ since each $O_n$ are dense. Therefore, we have found a metric space satisfying the Baire Category property that is not complete.
\end{proof}

%
% Example
%
{\bf Example:} Is $[0,1]$ countable? Since $[0,1]$ is closed and bounded we have that it must be compact. Since $[0,1]$ is compact we then have that it is totally bounded, and since total boundedness $\implies$ completeness we see that $[0,1]$ is complete. \\

Now, suppose $[0,1]$ is countable. Take $r \in [0,1]$ and look at the intersection
\begin{equation*}
	[0,1]\setminus\{r\}
\end{equation*}

\indent Since $\{r\}$ is closed we have $[0,1]\setminus\{r\}$ is dense and open. However, the intersection over all elements of $[0,1]$ is
\begin{equation*}
	\bigcap_{r\in[0,1]} [0,1]\setminus\{r\} = \emptyset
\end{equation*}

\indent If $[0,1]$ {\em were} countable then this intersection would have been a countable intersection of open sets. Therefore, since $[0,1]$ is also complete, we may use the Baire Category theorem to conclude that this countable union of dense open sets $[0,1]\setminus\{r\}$ would itself be dense. However, $\emptyset$ is clearly not dense in $[0,1]$ and so we must conclude that $[0,1]$ is, in fact, uncountable.

%
% Example
%
{\bf Example:} Consider the set of natural numbers $\N$. We know that $\N$ is complete since any Cauchy sequence will eventually have $\epsilon$ so small that $|x_n - x_m| < \epsilon$ implies $x_n = x_m$. Furthermore, $\N$ is countable, but the countable intersection from this complete space
\begin{equation*}
	\bigcap_{n\in\N} \N\setminus\{n\} = \emptyset
\end{equation*}

\indent This {\em does not} violate the Baire Category Theorem since the intersections $\N\setminus\{n\}$ is not dense because the closure $\overline{\N\setminus\{n\}} \neq \N$.






















































\end{document}