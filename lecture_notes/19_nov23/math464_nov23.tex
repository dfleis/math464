% --------------------------------------------------------------
% This is all preamble stuff that you don't have to worry about.
% Head down to where it says "Start here"
% --------------------------------------------------------------
 
\documentclass[12pt]{article}
 
\usepackage[margin=1in]{geometry} 
\usepackage{bm} % bold in mathmode \bm
\usepackage{amsmath,amsthm,amssymb,mathtools}
\usepackage{dsfont} % for indicator function \mathds 1
\usepackage{tikz,pgf,pgfplots}
\usepackage{enumerate} 
\usepackage[multiple]{footmisc} % for an adjascent footnote
\usepackage{graphicx,float} % figures
\usepackage{centernot} % for \centernot\implies (wrapped in \nimplies)

%% set noindent by default and define indent to be the standard indent length
\newlength\tindent
\setlength{\tindent}{\parindent}
\setlength{\parindent}{0pt}
\renewcommand{\indent}{\hspace*{\tindent}}

%% some math macros
\newcommand{\norm}[1]{\left\lVert#1\right\rVert} % vector norm
\newcommand*{\vv}[1]{\vec{\mkern0mu#1}} % \vec with arrow on top
\renewcommand{\Re}{\mathfrak {Re}}
\renewcommand{\Im}{\mathfrak {Im}}
\newcommand{\R}{\mathbb R}
\newcommand{\N}{\mathbb N}
\newcommand{\Z}{\mathbb Z}
\renewcommand{\P}{\mathbb P}
\newcommand{\Q}{\mathbb Q}
\newcommand{\E}{\mathbb E}
\newcommand{\F}{\mathbb F}
\newcommand{\C}{\mathbb C}
\newcommand{\X}{\mathbb X}
\newcommand{\powerset}{\mathcal P}
\renewcommand{\L}{\mathcal L}
\newcommand{\var}{\mathrm{Var}}
\newcommand{\Var}{\mathrm{Var}}
\newcommand{\cov}{\mathrm{Cov}}
\newcommand{\Cov}{\mathrm{Cov}}
\newcommand{\gm}{\mathrm{gm~}}
\newcommand{\am}{\mathrm{am~}}
\newcommand{\trace}{\mathrm{trace~}}
\newcommand{\Trace}{\mathrm{Trace~}}
\newcommand{\rank}{\mathrm{rank~}}
\newcommand{\Rank}{\mathrm{Rank~}}
\newcommand{\Span}{\mathrm{Span~}}
\newcommand{\card}{\mathrm{card}}
\newcommand{\Card}{\mathrm{Card}}
\newcommand{\limplies}{~\Longleftarrow ~} % leftwards implies, for some reason requires spacing to mirror the formatting of \implies (and therefore \rimplies below)
\newcommand{\rimplies}{\implies} % rightwards implies (for consistency)
\newcommand{\nimplies}{\centernot\implies} % rightwards implies with line struck through
\newcommand{\indist}{\,{\buildrel \mathcal D \over \sim}\,}
\newcommand\defeq{\mathrel{\stackrel{\makebox[0pt]{\mbox{\normalfont\tiny 
	def}}}{=}}} % equal sign with def above
\newcommand{\cl}{\mathrm{cl}} % closure of a set cl(X)
\newcommand{\Cl}{\mathrm{Cl}} % closure of a set Cl(X)
\newcommand{\closure}{\mathrm{closure}} % closure of a set closure(X)
\newcommand{\Closure}{\mathrm{Closure}} % closure of a set Closure(X)

\begin{document}
 
% --------------------------------------------------------------
%                         Start here
% --------------------------------------------------------------
 
\title{Real Analysis\\Lecture Notes}
\author{Metric Spaces}
\date{November 23 2016 \\ Last update: \today{}}
\maketitle

\section{Compactness}

\indent We wish to work up to the following result: A metric space is compact if and only if it is totally bounded and complete (where {\em complete} will be defined shortly). Note that this is similar to the Heine-Borel Theorem stating that a closed-bounded set is compact. However, in our case the added restrictions permit us to now state the result in both directions. \\

%
% Definition
%
{\bf Definition: {\em Complete metric space}.} We say that a metric space is \underline{complete} if every Cauchy sequence from $X$ is convergent. \\

Recall our previous result: If $X$ is a metric space then the following statements are equivalent
\begin{enumerate}
	\item $X$ is compact.
	\item $X$ satisfies the Bolzano-Weierstrass property.
	\item $X$ is sequentially compact (all sequences have a convergent subsequence). 
	\item $X$ is totally bounded (+ complete, to be shown)
\end{enumerate}

%
% Example
%
{\bf Example:} Consider $\N$, $\Q$, $\R$, and $\R$ under the discrete metric $\rho_d$
\begin{equation*}
	\rho_d(x,y) = 
	\begin{cases}
		1 & \text{if $x \neq y$} \\
		0 & \text{else}
	\end{cases}
\end{equation*}

Then
\begin{enumerate}
	\item $N$ is not bounded and so it cannot be totally bounded. However, $\N$ is complete since any Cauchy sequence from $\N$ will produce $0 < \epsilon$ so that our Cauchy sequence $(x_1, x_2, ...)$ will eventually repeat (and thus, be convergent) when $|x_n - x|$ gets small enough. 
	\item Likewise, $\R$ is not bounded, and so not totally bounded, but complete.
	\item $\Q$ is not bounded, and so not totally bounded. However, unlike $\N$ and $\R$, $\Q$ is {\em not} complete (recall the completeness axiom). Cauchy sequence from $\Q$ might {\em not} be convergent since the sequence $(x_n) \to \sqrt{2}$ will never converge to $\sqrt{2}$ in $\Q$ for all $\epsilon > 0$.
	\item $(0, 1)$ is totally bounded, but is {\em not} complete since we may have the Cauchy sequence $(x_n) \to 0$ or $(x_n) \to 1$, converging outside of our space.
	\item $\R$ under $\rho_d$ is complete since if $|x_m - x_n| < 1$ then $x_m = x_n$ and so $|x_m - x_n| = |x_m - x_m| = 0 < \epsilon$. However, $\R$ under the discrete metric is not totally bounded since the open cover
	\begin{equation*}
		\left\{ S_{x,\epsilon = \frac{1}{2}} \right\}_{x\in\R}
	\end{equation*}
	
	will not be able to cover $\R$ with only finitely many $S_{x,\epsilon}$.
\end{enumerate}

%
% Theorem
%
{\bf Theorem: {\em A metric space is compact if and only if it is totally bounded and complete}.}

\begin{proof} $(\implies)$ Suppose some metric space $X$ is compact. We immediately get that our space $X$ must be totally bounded since if $X$ is compact all covers have a finite subcover. Hence, the cover given by
\begin{equation*}
	X \subset \bigcup_{x\in X} S_{x,\epsilon}
\end{equation*}

has some finite subcover
\begin{equation*}
	X \subset \bigcup^n_{i = 1} S_{x_i,\epsilon}
\end{equation*}

satisfying the definition of total boundedness. We now seek to show that all Cauchy sequences are convergent. Let $(x_n)$ be some Cauchy sequence from $X$. We wish to prove that $(x_n) \to x$ for some $x \in X$. \\

Since $X$ is compact we are given that $X$ is {\em sequentially compact}. Therefore, all sequences from $X$ have a convergent subsequence. Therefore, our sequence $(x_n)$ has a subsequence $(x_{n_k})$ such that $(x_{n_k})\to x$ for some $x \in X$. \\

\indent However, since $(x_n)$ is a Cauchy sequence it must also converge to the same limit as its subsequences. We can see this by letting $(x_{n_k})$ be our subsequence and $x$ its limit. Then, 
\begin{equation*}
	\forall\,\epsilon > 0,~\exists,K\in\N,~\forall\,k\geq K,~ \left| x_{n_k} - x \right| < \epsilon
\end{equation*}

and if the sequence is Cauchy then for sufficiently large $N$ and $n,m \geq N$
\begin{equation*}
	|x_n - x_m| < \epsilon
\end{equation*}

Thus
\begin{equation*}
	\forall\,\epsilon > 0,~\exists,N\in\N,~\forall\,n\geq N,~ \left| x_{n} - x \right| < \epsilon
\end{equation*}

in particular, we may let $N = n_K$, and where $\forall\,n\geq N$ holds since for all $n, m \geq N$, $|x_n - x_m| < \epsilon$, i.e. all points beyond $N$ are at least as close as each other. \\

\indent Therefore, if $X$ is compact then $X$ is totally bounded, since the cover of open spheres has a finite subcover, and complete, since every Cauchy sequence is convergent, which completes the first direction \\


$(\limplies)$ Suppose some metric space $X$ is totally bounded and complete. To show that $X$ is complete it is sufficient to show that $X$ is sequentially compact. That is, it is sufficient to show that every sequence $(x_n)$ has a convergent subsequence. However, if we must show that every sequence $(x_n)$ has a convergent subsequence then it is sufficient to show that every sequence $(x_n)$ has a Cauchy subsequence $(x_{n_k})$. \\

\indent As an aside note that if some set $B$ is an infinite set and we conver it by finitely many sets $A_1, ..., A_n$ we must have that at least one of the $A_i$ is an infinite set. \\

\indent Now, consider the sequence $(x_n)$ and let $r = 1$ be a radius for a set of open spheres in $X$. Since $X$ is totally bounded $X$ must be covered by finitely many such spheres with radius $r = 1$. However, by our aside, at least one of our spheres must be an infinite set containing infinitely many such $x_i$. Label this infinite set by
\begin{equation*}
	S_{\epsilon = 1}
\end{equation*}

Pick one point from this spheres $x_{n_1} \in \{x_i\} = S_1$. Clearly the intersection
\begin{equation*}
	S_1 \setminus \{x_i\}
\end{equation*}

still has infinitely many elements left within it. \\

\indent Let $\epsilon = \frac{1}{2}$. Since $X$ is totally bounded $X$ is covered by finitely many spheres of radius $\frac{1}{2}$. However, these finitely many spheres which cover $X$ must also cover the intersection $S_1 \setminus \{x_i\}$. Therefore, there must exist some sphere $S_2$ with radius $\epsilon = \frac{1}{2}$ so that
\begin{equation*}
	S_2 \cap \left( S_1 \setminus \{x_i\} \right)
\end{equation*}

is still infinite. From this intersection pick the next point $x_{n_2}$
\begin{equation*}
	x_{n_2} \in S_2 \cap \left( S_1 \setminus \{x_i\} \right)
\end{equation*}

\indent Consider the distance $\rho(x_{n_1}, x_{n_2})$. Since both points are within $S_1$ by definition we must have a worst-case upper bound
\begin{equation*}
	\rho(x_{n_1}, x_{n_2}) < 2
\end{equation*}

With the same argument, let $r = \frac{1}{3}$ and construct $S_3$ a sphere of radius $\frac{1}{3}$ so that
\begin{equation*}
	S_3 \cap S_2 \cap S_1
\end{equation*}

has infinitely many points {\em beyond} the finite sequence 
\begin{equation*}
	\{x_1, x_2, ..., x_{n_1}, ..., x_{n_2}\}
\end{equation*}

\indent Pick one of these points beyond $x_{n_2}$ and call it $x_{n_3}$. Since $x_{n_3}$ and $x_{n_2}$ are both in $S_2$ with radius $\frac{1}{2}$ we have the worst-case upper bound of their distance given by
\begin{equation*}
	\rho(x_{n_1}, x_{n_2}) < 1
\end{equation*}

Continue inductively this manner. \\

\indent From this process we have constructed the sequence $\{x_{n_k}\}^\infty_{k = 1}$ which is a subsequence of $\{x_n\}^\infty_{n = 1}$ with distances
\begin{equation*}
	\rho(x_{n_k}, x_{n_l}) < \frac{2}{N}, \quad \text{if } k, l \geq N
\end{equation*}

\indent Thus $\rho(x_{n_k}, x_{n_l}) \to 0$ for sufficiently large $N$, and so our subsequence $(x_{n_k})$. Therefore, if $X$ is some totally bounded and complete metric space we have that every sequence from $X$ has a Cauchy subsequence, and so every sequence from $X$ has a convergent subsequence, and so $X$ is sequentially compact, and so $X$ is complete, as desired.
\end{proof}

%
% Aside
%
{\bf Aside:} In general, if $X$ is some metric space it can be {\em completed} via Cauchy sequences, and if $X$ is a totally bounded metric space its completion will also be totally bounded. \\

%
%
%
{\bf Baire Category Theorem:} Let $X$ be a complete metric space and take a countable family of dense open sets $\{O_i\}^\infty_{i = 1}$ from $X$. The intersection
\begin{equation*}
	\bigcap^\infty_{i = 1} O_i
\end{equation*}

is dense. \\

%
% Example
%
{\bf Example:} Let $X = \Q$ and take $q \in \Q$. Is the intersection $\Q \setminus \{q\}$ dense in $\Q$? Since $\{q\}$ is closed we must have that 
\begin{equation*}
	\Q \setminus \{q\} \quad \text{is open}
\end{equation*}

and
\begin{equation*}
	\bigcap_{q \in \Q} \left( \Q \setminus \{q\} \right) = \emptyset
\end{equation*}

but $\emptyset$ is trivially {\em not} dense in $\Q$! This is fine and doesn't contradict the Baire Category Theorem since $\Q$ is not a complete metric space. \\

%
% Corollary
%
{\bf Corollary: {\em (Consequence of the Baire Category Theorem)}.} We can show that $\Q$ is not a countable intersection of open sets. That is, $\Q$ is not a $G_\delta$.

\begin{proof} Suppose that $\Q$ is in fact a $G_\delta$ set so that
\begin{equation*}
	\Q = \bigcap^\infty_{i = 1} O_i
\end{equation*}

for $O_i$ open sets in $\R$. Since $\Q \subset O_i$ by definition, and $\Q$ is dense in $\R$, we have that each $O_i$ must also be dense in $\R$. \\

Now, consider the intersection
\begin{equation*}
	\left( \underbrace{\bigcap^\infty_{i = 1} O_i}_{=\Q} \right) \bigcap_{q \in Q} \left(\R \setminus \{q\} \right) = \emptyset
\end{equation*}

\indent However, this is a countable intersection of dense open sets in a complete space $\R$! Therefore, by the Baire Category Theorem this intersection {\em must be dense}, but the empty set $\emptyset$ is not dense! Therefore, $\Q$ is not a $G_\delta$.
\end{proof}

It turns out that the set of irrationals $\R\setminus\Q$ is a $G_\delta$ since 
\begin{align*}
	\Q &= \bigcup_{q\in\Q} \{q\} \\
	\implies \R\setminus\Q &= \R\setminus\bigcup_{q\in\Q} \{q\} \\
	&= \bigcap_{q\in\Q} \left( \R\setminus\{q\} \right)
\end{align*}

and this forms an intersection countably many open sets since there are countably many $q\in\Q$ and the singleton set $\{q\}$ is closed and so $\R\setminus\{q\}$ is open. 

Note that we began by writing $\Q$ as the countable union
\begin{equation*}
	\Q = \bigcup_{q\in\Q} \{q\}
\end{equation*}

so although $\Q$ is not a $G_\delta$, we have that $\Q$ is a $F_\sigma$. \\

%
% Example
%
{\bf Example:} Is the set of {\em irrationals} $\R\setminus\Q$ complete? (does every Cauchy sequence of irrationals converge? No!
	
\begin{proof} Consider the sequence $\left( 2 + \frac{1}{n\pi} \right)^\infty_{n = 1}$ taken from $\R\setminus\Q$. It is easy to see that our sequence does indeed converge and
\begin{equation*}
	\left( 2 + \frac{1}{n\pi} \right)^\infty_{n = 1} \longrightarrow 2
\end{equation*}
	
\indent Thus, we have found at least one Cauchy sequence taken from $\R\setminus\Q$ that does not converge within $\R\setminus\Q$ and so $\R\setminus\Q$ cannot be complete, as desired.
\end{proof}

%
% Definition
% 
{\bf Definition: {\em (Nowhere dense sets)}.} A subset $E$ of $X$, $E \subset X$, is said to be \underline{nowhere dense} in $X$ if the intersection
\begin{equation*}
	X\setminus\cl(E) \equiv X\setminus\overline{E} = X\cap\overline{E}^c
\end{equation*}

is dense in $X$. Alternatively, we can given an equivalent definition: A subset $E$ of $X$, $E \subset X$, is said to be \underline{nowhere dense} in $X$ if there is no nonempty open set $V$ that is a subset of the closure of $E$, $V \nsubseteq \overline{E}$. \\

\indent That is, $E$ is nowhere dense in $X$ if there is {\em no neighbourhood} in $X$ in which $E$ is dense. We can think of this as $E \subset X$ is nowhere dense in $X$ if it is {\em not dense} in any open subset of $X$. \\

%
% Example
%
{\bf Example:} Consider $X = \R$ and the singleton set $E = \{2\} \subset \R$. Clearly the closure of this set is
\begin{equation*}
	\overline{\{2\}} = \{2\}
\end{equation*}

Therefore, $X\setminus\overline{E}$ is given by
\begin{equation*}
	\R\setminus\{2\}
\end{equation*}

Then, to check if this intersection is dense in $X$ we take its closure
\begin{equation*}
	\overline{\R\setminus\{2\}} = \R
\end{equation*}

since $\R\setminus\{2\}$ is open with point $2$ removed. Thus, since $\overline{\R\setminus\{2\}} = \R$ we see that $X \setminus \overline{E} = \R\setminus\{2\}$ is indeed dense in $X = \R$, and so $E = \{2\}$ is nowhere dense in $X = \R$. \\

%
% Example
%
{\bf Example:} Let $X = \R$ and $E = \N$. We can see that the closure of $\N$ is $\overline{\N} = \N$. Thus, the intersection $X \setminus \overline{E}$ is
\begin{equation*}
	\R\setminus{\overline{\N}} = \R\setminus\N 
\end{equation*}

and so to check if $\R\setminus\N$ is dense in $\R$ we again take its closure
\begin{equation*}
	\overline{\R\setminus\N} = \R
\end{equation*}

Thus, $\R\setminus\N$ is dense in $\R$ and so $\N$ is nowhere dense in $\R$. \\

%
% Example
%
{\bf Example:} The set of rationals $\Q$ is {\em not} nowhere dense in $\R$ since
\begin{equation*}
	\R\setminus{\overline{\Q}} = \R\setminus\R = \emptyset
\end{equation*}

but
\begin{equation*}
	\overline{\R\setminus\overline{\Q}} = \overline{\R\setminus\R} = \overline{\emptyset} = \emptyset
\end{equation*}

and clearly $\emptyset$ is not dense in $\R$ and so $\Q$ is not nowhere dense in $\R$. \\

In general, no dense subset of some set will also be a nowhere dense set. \\

%
% Result
%
{\bf Result:} Suppose $O$ is some dense and open set in $X = \R$ and $E = \R\setminus O$. Since $O$ is an open set we have that $E = \R\setminus O$ must be closed so that $\overline{E} = E$. Therefore
\begin{align*}
	R\setminus\overline{E} &= R\setminus E \\
	&= \R\setminus \left( \R \setminus O \right) \\
	&= \R \cap \left( \R \setminus O \right)^c \\
	&= \R \cap \left( \R \cap O^c \right)^c \\
	&= \R \cap \left( \R^c \cup O \right) \\
	&= \R \cap \left( \emptyset \cup O \right) \\
	&= \R \cap \left( O \right) \\
	&= \R \cap O \\
	&= O
\end{align*}

However, $O$ is dense by assumption and $X\setminus\overline{E} = \R\setminus \left(\R\setminus O\right) = O$ is dense. Thus,
\begin{equation*}
	E = \R\setminus O
\end{equation*}

is nowhere dense. \\

%
% Definitions
% 
{\bf Definitions:} A subset $E$ of $X$ is called a \underline{meagre set} or \underline{of the first category} if it is a countable union of nowhere dense sets $E_n$
\begin{equation*}
	E = \bigcup^\infty_{n = 1} E_n \quad \text{$E_n$ nowhere dense}
\end{equation*}

For example, since $\Q$ can be written as
\begin{equation*}
	\Q = \bigcup_{q\in\Q} \{q\} 
\end{equation*}

where each $\{q\}$ is nowhere dense in $\R$. Thus, $\Q$ must be a meagre set/of the first category by definition. \\

\indent We say that a set is \underline{of the second category} if it is not of the first category, i.e. it cannot be expressed as a countable union of nowhere dense sets. \\

A set is \underline{residual} or \underline{comeagre} if it is the complement of a meagre set.



























































\end{document}