% --------------------------------------------------------------
% This is all preamble stuff that you don't have to worry about.
% Head down to where it says "Start here"
% --------------------------------------------------------------
 
\documentclass[12pt]{article}
 
\usepackage[margin=1in]{geometry} 
\usepackage{bm} % bold in mathmode \bm
\usepackage{amsmath,amsthm,amssymb,mathtools}
\usepackage{dsfont} % for indicator function \mathds 1
\usepackage{tikz,pgf,pgfplots}
\usepackage{enumerate} 
\usepackage[multiple]{footmisc} % for an adjascent footnote
\usepackage{graphicx,float} % figures
\usepackage{centernot} % for \centernot\implies (wrapped in \nimplies)

%% set noindent by default and define indent to be the standard indent length
\newlength\tindent
\setlength{\tindent}{\parindent}
\setlength{\parindent}{0pt}
\renewcommand{\indent}{\hspace*{\tindent}}

%% some math macros
\newcommand{\norm}[1]{\left\lVert#1\right\rVert} % vector norm
\newcommand*{\vv}[1]{\vec{\mkern0mu#1}} % \vec with arrow on top
\renewcommand{\Re}{\mathfrak {Re}}
\renewcommand{\Im}{\mathfrak {Im}}
\newcommand{\R}{\mathbb R}
\newcommand{\N}{\mathbb N}
\newcommand{\Z}{\mathbb Z}
\renewcommand{\P}{\mathbb P}
\newcommand{\Q}{\mathbb Q}
\newcommand{\E}{\mathbb E}
\newcommand{\F}{\mathbb F}
\newcommand{\C}{\mathbb C}
\newcommand{\X}{\mathbb X}
\newcommand{\powerset}{\mathcal P}
\renewcommand{\L}{\mathcal L}
\newcommand{\var}{\mathrm{Var}}
\newcommand{\Var}{\mathrm{Var}}
\newcommand{\cov}{\mathrm{Cov}}
\newcommand{\Cov}{\mathrm{Cov}}
\newcommand{\gm}{\mathrm{gm~}}
\newcommand{\am}{\mathrm{am~}}
\newcommand{\cl}{\mathrm{cl~}}
\newcommand{\trace}{\mathrm{trace~}}
\newcommand{\Trace}{\mathrm{Trace~}}
\newcommand{\rank}{\mathrm{rank~}}
\newcommand{\Rank}{\mathrm{Rank~}}
\newcommand{\Span}{\mathrm{Span~}}
\newcommand{\card}{\mathrm{card~}}
\newcommand{\Card}{\mathrm{Card~}}
\newcommand{\limplies}{~\Longleftarrow ~} % leftwards implies, for some reason requires spacing to mirror the formatting of \implies (and therefore \rimplies below)
\newcommand{\rimplies}{\implies} % rightwards implies (for consistency)
\newcommand{\nimplies}{\centernot\implies} % rightwards implies with line struck through
\newcommand{\indist}{\,{\buildrel \mathcal D \over \sim}\,}
\newcommand\defeq{\mathrel{\stackrel{\makebox[0pt]{\mbox{\normalfont\tiny 
	def}}}{=}}} % equal sign with def above
\renewcommand{\cl}{\mathrm{cl~}} % closure of a set cl(X)

\begin{document}
 
% --------------------------------------------------------------
%                         Start here
% --------------------------------------------------------------
 
\title{Real Analysis\\Lecture Notes}
\author{Metric Spaces}
\date{October 26 2016 \\ Last update: \today{}}
\maketitle

\section{Separable Metric Spaces}

\indent Last time we were talking about the Sorgenfry line, denoted by $\mathbb S$. The Sorgenfry line was defined to be a topology on $\R$ such that intervals of the form $[a,b)$ were said to be open. Furthermore, we had shown the result that $\mathbb S$ has a countable discrete subset.\footnote{Did we show this? I only see that we stated $\mathbb S$ has a countable dense subset $\Q$ such that $[a,b) \cap \Q \neq \emptyset$ so $\mathbb S$ must be separable, and that any compact\footnotemark subset of $\mathbb S$ is countable.}
\footnotetext{Compactness: All open covers have a finite subcover.} \\

\indent Finally, we ended last class with an incomplete proof that any compact subset of the Sorgenfry line $\mathcal S$ is countable. This was not true for $\R$ since, by the Heine-Borel theorem, the interval $[1,2]$ is compact since it is a closed bounded set. We present in this class a more complete proof of this statement. \\

%
% Claim
%
{\bf Claim: {\em Any compact subset of $\bm{\mathbb S}$ is countable}.} Let $\mathcal F$ be the family of intervals of the form
\begin{equation*}
	\mathcal F = \bigcup \left(-\infty, x - \frac{1}{n} \right), [x, \infty), \quad n \in \N
\end{equation*}

\indent We should be able to see that, over all $n \in \N$, the family $\mathcal F$ covers the real line $\R$. Therefore $\mathcal F$ must cover all subsets of $\R$. Now, suppose we take a \underline{compact} subset $C \subset \R$. For this compact set $C$ take some point $x \in C$. \\

\indent Our compact subset $C \subset \R$ must be covered by $\mathcal F$ since this family covers all subsets of $\R$. However, since $C$ is compact, there must be a finite subcover of $\mathcal F$ that covers $C$. What does this finite subset of $\mathcal F$ look like? \\

\indent Suppose our finite subcover contains the interval $[x,\infty)$. Then, in order to remain finite, $\mathcal F$ may only contain finitely many intervals of the form
\begin{equation*}
	\left(-\infty, \frac{1}{n_1} \right), \left(-\infty, \frac{1}{n_2} \right),... , \left(-\infty, \frac{1}{n_k} \right)
\end{equation*}

\indent An immediate consequence is that there must be some largest $n_p = \max \{n_1, n_2, ..., n_k\}$. For this largest $n_p$ we see that
\begin{align*}
	\frac{1}{n_p} &\leq \frac{1}{n_i}, \quad i = 1, 2, ..., k \\
	\implies x - \frac{1}{n_p} &\geq x - \frac{1}{n_i} \\
	\implies \left(-\infty, x - \frac{1}{n_i} \right) &\subset \left(-\infty, x - \frac{1}{n_p} \right)
\end{align*}

Thus, $C$ is covered by the union 
\begin{equation*}
	C \subset \left(-\infty, x - \frac{1}{n_p} \right) \cup [x, \infty)
\end{equation*}

Now, pick some number $a_x$ so that
\begin{equation*}
	a_x \in \left( x - \frac{1}{n_p}, x \right)
\end{equation*}

and consider the interval
\begin{equation*}
	(a_x, x] \cap C
\end{equation*}

Clearly, $x \in C$ by assumption, so $x \in (a_x, x] \cap C$, and since $C$ was covered by
\begin{equation*}
	C \subset \left(-\infty, x - \frac{1}{n_p} \right) \cup [x,\infty)
\end{equation*}

Therefore, since $C \cap \left[x - \frac{1}{n_p}, x\right) = \emptyset$,
\begin{equation*}
	(a_x, x] \cap C = \{x\}
\end{equation*}

Suppose we repeat this process and argument for some different $x' \in C$, $x' \neq x$. Then
\begin{equation*}
	(a_{x'}, x'] \cap C = \{x'\}
\end{equation*}

but $x \leq a_{x'}$,\footnote{Where does this inequality come from? Is this an assumption/criteria for selecting $x'$ or is this a consequence of something else?} hence
\begin{equation*}
	(a_x, x] \cap (a_{x'}, x'] = \emptyset
\end{equation*}

Therefore, the intervals $(a_x, x]$ for $x \in C$ are pairwise disjoint. However, $a_x \in \left(x - \frac{1}{n_p}, x\right) \implies a_x < x$, hence
\begin{equation*}
	\exists\,q_x \in \Q, ~ a_x < q_x < x
\end{equation*}

\indent Can two different $x$'s have the same rational $q_x$? No! Since our intervals $(a_x, x]$ are pairwise disjoint we know that each $q_x$ must uniquely identify a single interval. Since the rationals $\Q$ are countable we may conclude that there are only countably many intervals of the form $(a_x, x]$. \\

\indent However, every $x$ generates a unique interval $(a_x, x]$! Therefore, there are only countably many points $x \in C$. That is, our set $C$, a compact subset of $\R$ is countable, and since $C$ was arbitrary we may conclude that any compact subset of $\mathbb S$ is indeed countable, as desired. 

\section{Topological Bases}

%
% Definition: Base for a topology
%
{\bf Definition: {\em (Base for a topology)}} We say that a \underline{base} for a topology $(X, \rho)$ is a family of open sets $B$ such that every open set is the union of open sets in $B$. That is, for any open set $O \subset X$, a base $B$ composed of open sets $B_i$, $B = \{B_i\}_{i\in I}$, satisfies 
\begin{equation*}
	O = \bigcup_{i \in I} B_i
\end{equation*}

%
% Example
%
{\bf Example:} Consider $B$ a base for the topology on the Sorgenfry line $\mathbb S$. Take point $x \in \R$. Then set $[x, x + 1)$ is open in $\mathbb S$ is open by definition. Therefore, under our base $B$ we have
\begin{equation*}
	\exists\, B_x \in B \text{ such that } x \in B_x \subset [x, x + 1)
\end{equation*}

\indent Take $y \neq x$. Without loss of generality let $x < y$. From $[y, y + 1)$ we get an open set $B_y$ such that
\begin{equation*}
	y \in B_y \subset [y, y + 1)
\end{equation*}

\indent  Since $x < y$ we know that $x \notin B_y \subset [y, y + 1)$. So, $x \in B_x, y \in B_y$ and since $x \neq y$ we have $B_x \neq B_y$. This gives us that a mapping from $\R \to B$ is one-to-one. Therefore, the image of $\R$ must have cardinality greater than to $\R$ itself. However, we know that $\R$ is uncountable. Hence,
\begin{equation*}
	\card B \geq \card \R
\end{equation*}

\indent Therefore, $B$ must be uncountable. However, we found last time that if $X$ is a metric space then the following are equivalent:
\begin{enumerate}
	\item $X$ is separable.
	\item $X$ has a countable base.
\end{enumerate}

but for the Sorgenfry $\mathbb S$ we have that
\begin{enumerate}
	\item $\mathbb S$ is separable.
	\item Any base for $\mathbb S$ is uncountable.
\end{enumerate}

\indent Therefore, we must conclude that $\mathbb S$ is not {\em metrizable}. That is, there is no metric on $\R$ for which the open sets generated by the metric are exactly the open sets in the Sorgenfry line. \\

%
% Definition: Equivalent metric spaces
%
{\bf Definition: {\em (Equivalent metric spaces)}} Let $X$ be some space with metrics $\rho$ and $\rho'$. We say that metrics $\rho$ and $\rho'$ are \underline{equivalent} if they both give rise to the same open sets. \\

{\bf [BOOK DEFINITION OF EQUIVALENT METRIC SPACES]} \\


%
% Example
%
{\bf Example:} Consider $X = \R$ and 
\begin{align*}
	\rho(x,y) &= |y - x| \\
	\rho(x,y) &= \frac{1}{2}|y - x|
\end{align*}

Clearly any open set defined by the metric spaces $(X, \rho)$ can also be found in $(X, \rho')$. \\

%
% Definition: Bounded metric spaces
%
{\bf Definition: {\em (Bounded metrics)}} A metric $\rho$ is said to be \underline{bounded} in space $X$ if, for $M \in \R$ finite,
\begin{equation*}
	\forall\,x,y\in X,~\rho(x,y) \leq M
\end{equation*}


An immediate example is given by the discrete metric
\begin{equation*}
	\rho_d(x,y) = 
	\begin{cases}
		1 & \text{if } x \neq y \\
		0 & \text{if } x = y
	\end{cases}
\end{equation*}

then, for all $x,y \in X$ we find $\rho_d(x,y) \leq 1$. \\

%
% Proposition: Ever metric is equivalent to some bounded metric
%
{\bf Proposition: {\em Every metric is equivalent to some bounded metric}}. In particular, if $(X, \rho)$ is some metric space then $\rho$ will be equivalent to 
\begin{equation*}
	\frac{\rho(x,y)}{1 + \rho(x,y)} = \rho'(x,y)
\end{equation*}

which will be shown to be bound by $\rho'(x,y) \leq 1$ and give rise to the same open sets generated by $\rho$.

\begin{proof} \hfill\\
\begin{enumerate}
	\item Is $\rho'$ bounded? Clearly $\rho' \geq 0$ by definition, and as $\rho \to \infty$ we see that
	\begin{equation*}
		\rho' = \frac{\rho}{1 + \rho} \to 1 \implies \rho' \in [0, 1)
	\end{equation*}
	
	Hence, $\rho'(x, y) \leq 1$ for all $x,y \in X$.
	
	\item Given that $\rho$ is a metric, is $\rho'$ also a metric?
	\begin{enumerate}
		\item Clearly $\rho'(x,y) \geq 0$.
		
		\item If $\rho' = \frac{\rho}{1 + \rho} = 0$ then $\rho = 0 \iff x = y$. Thus,
		$\rho' = 0 \iff x = y$.
		
		\item $\rho'(x,y) = \rho'(y,x)$ since $\rho(x,y) = \rho(y, x)$. \\

		\item Does the triangle inequality hold? Working backwards we wish to prove that
		\begin{equation*}
			\rho'(x,z) \leq \rho'(x,y) + \rho'(y,z) 
		\end{equation*}
		
		but $\rho' = \frac{\rho}{1 + \rho}$, so
		\begin{align*}
			\frac{\rho(x,z)}{1 + \rho(x,z)} &\leq \frac{\rho(x,y)}{1 + \rho(x,y)} + \frac{\rho(y,z)}{1 + \rho(y,z)} \\
			\frac{\rho(x,z)}{1 + \rho(x,z)} &\leq \frac{ \rho(x,y) + \rho(x,y)\rho(y,z) + \rho(y,z) + \rho(y,z)\rho(x,y) }{ (1 + \rho(x,y))(1 + \rho(y,z)) } \\
			&\vdots
		\end{align*}
		
		Continuous this tedious algebra we find that the triangle inequality is indeed satisfied.
	\end{enumerate}
	
	\item Are the metrics $\rho$ and $\rho' = \frac{\rho}{1 + \rho}$ equivalent? Recall that open sets under a metric $\rho$ are unions of spheres of the form
	\begin{equation*}
		B_{x,\epsilon,\rho} = \{y\in X ~:~ \rho(x,y) < \epsilon\}
	\end{equation*}
	
	\indent Therefore, we will have equivalence between $\rho$ and $\rho'$ if the spheres generated by $\rho$ are also open in the topology generated by $\rho'$, and if the spheres generated by $\rho'$ are open in the topology of $\rho$. \\
	
	Before moving on to prove this result we must quickly prove the following lemma:
	
	%
	% Lemma
	%
	{\bf Lemma:}
	\begin{enumerate}
		\item $B_{x,r,\rho} \subseteq B_{x,r,\rho'}$ ($B$ centered at $x$ with radius $r$ under $\rho$).
		
		\begin{proof} Let $y \in B_{x,r,\rho}$. Then
		\begin{align*}
			\rho(x,y) &< r \\
			\implies \frac{\rho(x,y)}{\rho(x,y) + 1} &< r \quad \text{(since $1 \leq 1 + \rho(x,y)$)} \\
			\iff& \rho'(x,y) < r \\
			\implies y &\in B_{x,r,\rho'} \\
			\implies B_{x,r,\rho} &\subseteq B_{x,r,\rho'}
		\end{align*}
		\end{proof}
		
		\item $B_{x,\frac{r}{r + 1},\rho'} \subseteq B_{x,r,\rho}$.
		
		\begin{proof} Take $z \in B_{x, \frac{r}{r + 1},\rho'}$. Then
		\begin{align*}
			\rho'(x,z) &< \frac{r}{r + 1} \\
			\implies \frac{\rho(x,z)}{\rho(x,z) + 1} &< \frac{r}{r + 1} \\
			\implies \rho(x,z) (r + 1) &< r (\rho(x,z) + 1) \\
			\implies \rho(x,z) r + \rho(x,z) &< r\rho(x,z) + r \\
			\implies \rho(x,z) &< r \\
			\implies z &\in B_{x,r,\rho} \\
			\implies B_{x,\frac{r}{r + 1},\rho'} &\subseteq B_{x,r,\rho}
		\end{align*}
		\end{proof}
	\end{enumerate}
	
	\indent Now, back to our proposition at hand. Are the open sets under $\rho$ open under $\rho'$ and vice-versa? In particular, is $B_{x,r,\rho'}$ open in the $\rho$-topology? Take some point $t \in B_{x,r,\rho'}$. Since $B_{x,r,\rho'}$ is open we can find a smaller sphere lying within it:
	\begin{equation*}
		\exists\,d > 0 \text{ such that } B_{t,d,\rho'} \subset B_{x,r,\rho'}
	\end{equation*}
	
	and using our first lemma:
	\begin{equation*}
		B_{t,d,\rho} \subset B_{t,d,\rho'}
	\end{equation*}
	
	\indent Therefore, every point $t \in B_{x,r,\rho'}$ lies within some open set in the $\rho$ metric given by $B_{t,d,\rho}$. That is, every point lies within some open subset $B_{t,d,\rho}$ of $B_{x,r,\rho'}$. \\
	
	\indent Therefore, taking the union of all $t \in B_{x,r,\rho'}$ we find that the open set $B_{x,r,\rho'}$ is a union of open sets of the form $B_{t,d,\rho}$, which are open in the $\rho$-topology. \\
	
	\indent By the same argument we may find that $B_{x,r,\rho}$ will be open in the $\rho'$-topology. Hence, every metric $\rho$ is equivalent to a bounded metric $\rho' = \frac{\rho}{1 + \rho}$.
\end{enumerate}
\end{proof}



















































\end{document}